% Options for packages loaded elsewhere
\PassOptionsToPackage{unicode}{hyperref}
\PassOptionsToPackage{hyphens}{url}
%
\documentclass[
]{article}
\usepackage{amsmath,amssymb}
\usepackage{iftex}
\ifPDFTeX
  \usepackage[T1]{fontenc}
  \usepackage[utf8]{inputenc}
  \usepackage{textcomp} % provide euro and other symbols
\else % if luatex or xetex
  \usepackage{unicode-math} % this also loads fontspec
  \defaultfontfeatures{Scale=MatchLowercase}
  \defaultfontfeatures[\rmfamily]{Ligatures=TeX,Scale=1}
\fi
\usepackage{lmodern}
\ifPDFTeX\else
  % xetex/luatex font selection
\fi
% Use upquote if available, for straight quotes in verbatim environments
\IfFileExists{upquote.sty}{\usepackage{upquote}}{}
\IfFileExists{microtype.sty}{% use microtype if available
  \usepackage[]{microtype}
  \UseMicrotypeSet[protrusion]{basicmath} % disable protrusion for tt fonts
}{}
\makeatletter
\@ifundefined{KOMAClassName}{% if non-KOMA class
  \IfFileExists{parskip.sty}{%
    \usepackage{parskip}
  }{% else
    \setlength{\parindent}{0pt}
    \setlength{\parskip}{6pt plus 2pt minus 1pt}}
}{% if KOMA class
  \KOMAoptions{parskip=half}}
\makeatother
\usepackage{xcolor}
\usepackage[margin=1in]{geometry}
\usepackage{color}
\usepackage{fancyvrb}
\newcommand{\VerbBar}{|}
\newcommand{\VERB}{\Verb[commandchars=\\\{\}]}
\DefineVerbatimEnvironment{Highlighting}{Verbatim}{commandchars=\\\{\}}
% Add ',fontsize=\small' for more characters per line
\usepackage{framed}
\definecolor{shadecolor}{RGB}{248,248,248}
\newenvironment{Shaded}{\begin{snugshade}}{\end{snugshade}}
\newcommand{\AlertTok}[1]{\textcolor[rgb]{0.94,0.16,0.16}{#1}}
\newcommand{\AnnotationTok}[1]{\textcolor[rgb]{0.56,0.35,0.01}{\textbf{\textit{#1}}}}
\newcommand{\AttributeTok}[1]{\textcolor[rgb]{0.13,0.29,0.53}{#1}}
\newcommand{\BaseNTok}[1]{\textcolor[rgb]{0.00,0.00,0.81}{#1}}
\newcommand{\BuiltInTok}[1]{#1}
\newcommand{\CharTok}[1]{\textcolor[rgb]{0.31,0.60,0.02}{#1}}
\newcommand{\CommentTok}[1]{\textcolor[rgb]{0.56,0.35,0.01}{\textit{#1}}}
\newcommand{\CommentVarTok}[1]{\textcolor[rgb]{0.56,0.35,0.01}{\textbf{\textit{#1}}}}
\newcommand{\ConstantTok}[1]{\textcolor[rgb]{0.56,0.35,0.01}{#1}}
\newcommand{\ControlFlowTok}[1]{\textcolor[rgb]{0.13,0.29,0.53}{\textbf{#1}}}
\newcommand{\DataTypeTok}[1]{\textcolor[rgb]{0.13,0.29,0.53}{#1}}
\newcommand{\DecValTok}[1]{\textcolor[rgb]{0.00,0.00,0.81}{#1}}
\newcommand{\DocumentationTok}[1]{\textcolor[rgb]{0.56,0.35,0.01}{\textbf{\textit{#1}}}}
\newcommand{\ErrorTok}[1]{\textcolor[rgb]{0.64,0.00,0.00}{\textbf{#1}}}
\newcommand{\ExtensionTok}[1]{#1}
\newcommand{\FloatTok}[1]{\textcolor[rgb]{0.00,0.00,0.81}{#1}}
\newcommand{\FunctionTok}[1]{\textcolor[rgb]{0.13,0.29,0.53}{\textbf{#1}}}
\newcommand{\ImportTok}[1]{#1}
\newcommand{\InformationTok}[1]{\textcolor[rgb]{0.56,0.35,0.01}{\textbf{\textit{#1}}}}
\newcommand{\KeywordTok}[1]{\textcolor[rgb]{0.13,0.29,0.53}{\textbf{#1}}}
\newcommand{\NormalTok}[1]{#1}
\newcommand{\OperatorTok}[1]{\textcolor[rgb]{0.81,0.36,0.00}{\textbf{#1}}}
\newcommand{\OtherTok}[1]{\textcolor[rgb]{0.56,0.35,0.01}{#1}}
\newcommand{\PreprocessorTok}[1]{\textcolor[rgb]{0.56,0.35,0.01}{\textit{#1}}}
\newcommand{\RegionMarkerTok}[1]{#1}
\newcommand{\SpecialCharTok}[1]{\textcolor[rgb]{0.81,0.36,0.00}{\textbf{#1}}}
\newcommand{\SpecialStringTok}[1]{\textcolor[rgb]{0.31,0.60,0.02}{#1}}
\newcommand{\StringTok}[1]{\textcolor[rgb]{0.31,0.60,0.02}{#1}}
\newcommand{\VariableTok}[1]{\textcolor[rgb]{0.00,0.00,0.00}{#1}}
\newcommand{\VerbatimStringTok}[1]{\textcolor[rgb]{0.31,0.60,0.02}{#1}}
\newcommand{\WarningTok}[1]{\textcolor[rgb]{0.56,0.35,0.01}{\textbf{\textit{#1}}}}
\usepackage{longtable,booktabs,array}
\usepackage{calc} % for calculating minipage widths
% Correct order of tables after \paragraph or \subparagraph
\usepackage{etoolbox}
\makeatletter
\patchcmd\longtable{\par}{\if@noskipsec\mbox{}\fi\par}{}{}
\makeatother
% Allow footnotes in longtable head/foot
\IfFileExists{footnotehyper.sty}{\usepackage{footnotehyper}}{\usepackage{footnote}}
\makesavenoteenv{longtable}
\usepackage{graphicx}
\makeatletter
\def\maxwidth{\ifdim\Gin@nat@width>\linewidth\linewidth\else\Gin@nat@width\fi}
\def\maxheight{\ifdim\Gin@nat@height>\textheight\textheight\else\Gin@nat@height\fi}
\makeatother
% Scale images if necessary, so that they will not overflow the page
% margins by default, and it is still possible to overwrite the defaults
% using explicit options in \includegraphics[width, height, ...]{}
\setkeys{Gin}{width=\maxwidth,height=\maxheight,keepaspectratio}
% Set default figure placement to htbp
\makeatletter
\def\fps@figure{htbp}
\makeatother
\setlength{\emergencystretch}{3em} % prevent overfull lines
\providecommand{\tightlist}{%
  \setlength{\itemsep}{0pt}\setlength{\parskip}{0pt}}
\setcounter{secnumdepth}{-\maxdimen} % remove section numbering
\ifLuaTeX
  \usepackage{selnolig}  % disable illegal ligatures
\fi
\usepackage{bookmark}
\IfFileExists{xurl.sty}{\usepackage{xurl}}{} % add URL line breaks if available
\urlstyle{same}
\hypersetup{
  pdftitle={Analiza zbioru danych zawierającego informacje o winach pochodzących z północy Portugalii},
  hidelinks,
  pdfcreator={LaTeX via pandoc}}

\title{Analiza zbioru danych zawierającego informacje o winach
pochodzących z północy Portugalii}
\author{}
\date{\vspace{-2.5em}}

\begin{document}
\maketitle

Zbiór danych wykorzystany do poniższej analizy pochodzi ze strony
\href{https://www.kaggle.com/datasets/shelvigarg/wine-quality-dataset/data}{kaggle.com}.
Jest to zbiór zawierający dane o czerwonym oraz białym wariancie wina
``Vinho Verde'', pochodzącego z północy Portugalii.

Celem niniejszej analizy jest zastosowanie \emph{regresji liniowej}.

\emph{Regresja liniowa} to metoda statystyczna, wykorzystywana do
badania zależności między jedną zmienną zależną a jedną lub większą
liczbą zmiennych niezależnych. Polega ona na próbie dopasowania linii do
danych, aby zrozumieć charakter relacji między nimi.

Załadujemy na początek potrzebne biblioteki, żeby funkcje, które są
wykorzystywane w projekcie, działały poprawnie.

\begin{Shaded}
\begin{Highlighting}[]
\FunctionTok{library}\NormalTok{(car)}
\FunctionTok{library}\NormalTok{(corrplot)}
\FunctionTok{library}\NormalTok{(dplyr)}
\FunctionTok{library}\NormalTok{(faraway)}
\FunctionTok{library}\NormalTok{(lmtest)}
\FunctionTok{library}\NormalTok{(MASS)}
\FunctionTok{library}\NormalTok{(nortest)}
\FunctionTok{library}\NormalTok{(RColorBrewer)}
\end{Highlighting}
\end{Shaded}

\section{Dane - ich struktura oraz
klasyfikacja}\label{dane---ich-struktura-oraz-klasyfikacja}

Załadujmy plik winequalityN.csv oraz prześledźmy jego strukturę danych.

\begin{Shaded}
\begin{Highlighting}[]
\NormalTok{wine }\OtherTok{\textless{}{-}} \FunctionTok{read.csv}\NormalTok{(}\StringTok{"C:}\SpecialCharTok{\textbackslash{}\textbackslash{}}\StringTok{Users}\SpecialCharTok{\textbackslash{}\textbackslash{}}\StringTok{Nikola}\SpecialCharTok{\textbackslash{}\textbackslash{}}\StringTok{Documents}\SpecialCharTok{\textbackslash{}\textbackslash{}}\StringTok{Nikola Chmielewska}\SpecialCharTok{\textbackslash{}\textbackslash{}}\StringTok{R}\SpecialCharTok{\textbackslash{}\textbackslash{}}\StringTok{Datasets}\SpecialCharTok{\textbackslash{}\textbackslash{}}\StringTok{winequalityN.csv"}\NormalTok{)}
\FunctionTok{str}\NormalTok{(wine)}
\end{Highlighting}
\end{Shaded}

\begin{verbatim}
## 'data.frame':    6497 obs. of  13 variables:
##  $ type                : chr  "white" "white" "white" "white" ...
##  $ fixed.acidity       : num  7 6.3 8.1 7.2 7.2 8.1 6.2 7 6.3 8.1 ...
##  $ volatile.acidity    : num  0.27 0.3 0.28 0.23 0.23 0.28 0.32 0.27 0.3 0.22 ...
##  $ citric.acid         : num  0.36 0.34 0.4 0.32 0.32 0.4 0.16 0.36 0.34 0.43 ...
##  $ residual.sugar      : num  20.7 1.6 6.9 8.5 8.5 6.9 7 20.7 1.6 1.5 ...
##  $ chlorides           : num  0.045 0.049 0.05 0.058 0.058 0.05 0.045 0.045 0.049 0.044 ...
##  $ free.sulfur.dioxide : num  45 14 30 47 47 30 30 45 14 28 ...
##  $ total.sulfur.dioxide: num  170 132 97 186 186 97 136 170 132 129 ...
##  $ density             : num  1.001 0.994 0.995 0.996 0.996 ...
##  $ pH                  : num  3 3.3 3.26 3.19 3.19 3.26 3.18 3 3.3 3.22 ...
##  $ sulphates           : num  0.45 0.49 0.44 0.4 0.4 0.44 0.47 0.45 0.49 0.45 ...
##  $ alcohol             : num  8.8 9.5 10.1 9.9 9.9 10.1 9.6 8.8 9.5 11 ...
##  $ quality             : int  6 6 6 6 6 6 6 6 6 6 ...
\end{verbatim}

\begin{Shaded}
\begin{Highlighting}[]
\FunctionTok{unique}\NormalTok{(wine}\SpecialCharTok{$}\NormalTok{type)}
\end{Highlighting}
\end{Shaded}

\begin{verbatim}
## [1] "white" "red"
\end{verbatim}

Na podstawie powyższej komendy widzimy, że zbiór danych zawiera również
informacje o wariancie wina: czerwonym i białym.

Dokonajmy jeszcze objaśnienia zmiennych występujących w zbiorze danych.

\begin{longtable}[]{@{}ll@{}}
\toprule\noalign{}
Nazwa & Opis \\
\midrule\noalign{}
\endhead
\bottomrule\noalign{}
\endlastfoot
type & rodzaj wina (białe, czerwone) \\
fixed.acidity & stała kwasowość \\
volatile acidity & zmienna kwasowość \\
citric acid & kwas cytrynowy \\
residual sugar & resztowy cukier \\
chlorides & chlorki \\
free sulfur dioxide & wolny dwutlenek siarki \\
total sulfur dioxide & całkowity dwutlenek siarki \\
density & gęstość \\
pH -- potential of hydrogen & współczynnik kwasowości/zasadowości pH \\
sulphates & siarczyny \\
alcohol & alkohol \\
quality & jakość \\
\end{longtable}

Oraz, aby lepiej zrozumieć dane, zobaczmy ich podsumowanie:

\begin{Shaded}
\begin{Highlighting}[]
\FunctionTok{summary}\NormalTok{(wine)}
\end{Highlighting}
\end{Shaded}

\begin{verbatim}
##      type           fixed.acidity    volatile.acidity  citric.acid    
##  Length:6497        Min.   : 3.800   Min.   :0.0800   Min.   :0.0000  
##  Class :character   1st Qu.: 6.400   1st Qu.:0.2300   1st Qu.:0.2500  
##  Mode  :character   Median : 7.000   Median :0.2900   Median :0.3100  
##                     Mean   : 7.217   Mean   :0.3397   Mean   :0.3187  
##                     3rd Qu.: 7.700   3rd Qu.:0.4000   3rd Qu.:0.3900  
##                     Max.   :15.900   Max.   :1.5800   Max.   :1.6600  
##                     NA's   :10       NA's   :8        NA's   :3       
##  residual.sugar     chlorides       free.sulfur.dioxide total.sulfur.dioxide
##  Min.   : 0.600   Min.   :0.00900   Min.   :  1.00      Min.   :  6.0       
##  1st Qu.: 1.800   1st Qu.:0.03800   1st Qu.: 17.00      1st Qu.: 77.0       
##  Median : 3.000   Median :0.04700   Median : 29.00      Median :118.0       
##  Mean   : 5.444   Mean   :0.05604   Mean   : 30.53      Mean   :115.7       
##  3rd Qu.: 8.100   3rd Qu.:0.06500   3rd Qu.: 41.00      3rd Qu.:156.0       
##  Max.   :65.800   Max.   :0.61100   Max.   :289.00      Max.   :440.0       
##  NA's   :2        NA's   :2                                                 
##     density             pH          sulphates         alcohol     
##  Min.   :0.9871   Min.   :2.720   Min.   :0.2200   Min.   : 8.00  
##  1st Qu.:0.9923   1st Qu.:3.110   1st Qu.:0.4300   1st Qu.: 9.50  
##  Median :0.9949   Median :3.210   Median :0.5100   Median :10.30  
##  Mean   :0.9947   Mean   :3.218   Mean   :0.5312   Mean   :10.49  
##  3rd Qu.:0.9970   3rd Qu.:3.320   3rd Qu.:0.6000   3rd Qu.:11.30  
##  Max.   :1.0390   Max.   :4.010   Max.   :2.0000   Max.   :14.90  
##                   NA's   :9       NA's   :4                       
##     quality     
##  Min.   :3.000  
##  1st Qu.:5.000  
##  Median :6.000  
##  Mean   :5.818  
##  3rd Qu.:6.000  
##  Max.   :9.000  
## 
\end{verbatim}

Sprawdźmy jeszcze, czy w zbiorze wszystkie dane są podane:

\begin{Shaded}
\begin{Highlighting}[]
\NormalTok{(}\FunctionTok{sapply}\NormalTok{(wine, }\ControlFlowTok{function}\NormalTok{(x) \{}\FunctionTok{sum}\NormalTok{(}\FunctionTok{is.na}\NormalTok{(x))\}))}
\end{Highlighting}
\end{Shaded}

\begin{verbatim}
##                 type        fixed.acidity     volatile.acidity 
##                    0                   10                    8 
##          citric.acid       residual.sugar            chlorides 
##                    3                    2                    2 
##  free.sulfur.dioxide total.sulfur.dioxide              density 
##                    0                    0                    0 
##                   pH            sulphates              alcohol 
##                    9                    4                    0 
##              quality 
##                    0
\end{verbatim}

Jak widać powyżej, nasza tabela zawiera wartości NA. Usuniemy zatem
wiersze, które nie posiadają danych, ponieważ w kontekście całego zbioru
danych jest to marginalna ilość. Robimy to, ponieważ brak danych może
obniżyć jakość późniejszej analizy danych.

\begin{Shaded}
\begin{Highlighting}[]
\NormalTok{wine\_clean }\OtherTok{\textless{}{-}} \FunctionTok{na.omit}\NormalTok{(wine) }\CommentTok{\#usuwa wiersze zawierające wartości NA}
\end{Highlighting}
\end{Shaded}

Wyświetlmy jeszcze raz nasz nowy zbiór danych z usuniętymi wartościami
NA.

\begin{Shaded}
\begin{Highlighting}[]
\NormalTok{(}\FunctionTok{sapply}\NormalTok{(wine\_clean, }\ControlFlowTok{function}\NormalTok{(x) \{}\FunctionTok{sum}\NormalTok{(}\FunctionTok{is.na}\NormalTok{(x))\}))}
\end{Highlighting}
\end{Shaded}

\begin{verbatim}
##                 type        fixed.acidity     volatile.acidity 
##                    0                    0                    0 
##          citric.acid       residual.sugar            chlorides 
##                    0                    0                    0 
##  free.sulfur.dioxide total.sulfur.dioxide              density 
##                    0                    0                    0 
##                   pH            sulphates              alcohol 
##                    0                    0                    0 
##              quality 
##                    0
\end{verbatim}

Jak widzimy, w tej chwili żadna zmienna nie jest obarczona brakiem
danych, więc nasz zbiór danych jest gotowy do obróbki.

Jednocześnie w powyższym zestawie danych zmienną zależną jest atrybut
quality, a więc ocena jakości wina. Pozostałe zmienne są zmiennymi
niezależnymi, ponieważ wpływają one na końcową ocenę wina; są to również
zmienne ilościowe. Natomiast zmienną objaśnianą `quality' można
potraktować zarówno jako zmienną ilościową oraz jako zmienną nominalną o
uporządkowanych kategoriach (jakościową).

W niniejszej pracy zmienną `quality' potraktujemy jako zmienną
ilościową, rozważając modele regresji liniowej. Jednak podczas tworzenia
modelu proporcjonalnych szans, zmienną `quality' potraktujemy jako
zmienną jakościową. Warto w tym miejscu zaznaczyć, że w celu weryfikacji
hipotez będziemy zakładać poziom istotności \emph{α = 0,05}.

\subsection{Podział zbioru danych}\label{podziaux142-zbioru-danych}

Podzielimy nasz zbiór danych na podstawie wariantu wina - białe i
czerwone. W niniejszej analizie skupimy się jedynie na przeanalizowaniu
danych dot. białego wina. Dla czerwonego wariantu wina można zrobić
osobną analizę, przechodząc analogicznie przez kroki, jakie wykonaliśmy
dla białego wina.

\begin{Shaded}
\begin{Highlighting}[]
\NormalTok{white\_wine }\OtherTok{\textless{}{-}} \FunctionTok{subset}\NormalTok{(wine\_clean, type }\SpecialCharTok{==} \StringTok{"white"}\NormalTok{)}
\end{Highlighting}
\end{Shaded}

Dodatkowo usuniemy kolumny tekstowe, ponieważ tylko kolumny z danymi
numerycznymi będą nam potrzebne do dalszej analizy.

\begin{Shaded}
\begin{Highlighting}[]
\NormalTok{white\_numeric }\OtherTok{\textless{}{-}}\NormalTok{ white\_wine[}\FunctionTok{sapply}\NormalTok{(white\_wine, is.numeric)]}
\end{Highlighting}
\end{Shaded}

Stworzymy teraz zmienne pomocnicze, które pomogą nam podzielić zbiór
danych na 3 podzbiory.

\begin{Shaded}
\begin{Highlighting}[]
\NormalTok{N }\OtherTok{\textless{}{-}} \FunctionTok{nrow}\NormalTok{(white\_numeric)}
\NormalTok{I }\OtherTok{\textless{}{-}}\NormalTok{ (}\DecValTok{1}\SpecialCharTok{:}\NormalTok{N)}
\end{Highlighting}
\end{Shaded}

Teraz losujemy indeksy.

\begin{Shaded}
\begin{Highlighting}[]
\FunctionTok{set.seed}\NormalTok{(}\DecValTok{300}\NormalTok{)}
\NormalTok{I\_l }\OtherTok{\textless{}{-}} \FunctionTok{sample.int}\NormalTok{(N, }\AttributeTok{size =} \FunctionTok{round}\NormalTok{(N}\SpecialCharTok{/}\DecValTok{2}\NormalTok{)) }\CommentTok{\#50\% danych}
\NormalTok{I\_v }\OtherTok{\textless{}{-}} \FunctionTok{sample}\NormalTok{(}\FunctionTok{setdiff}\NormalTok{(I, I\_l), }\AttributeTok{size =} \FunctionTok{round}\NormalTok{(N}\SpecialCharTok{/}\DecValTok{4}\NormalTok{)) }\CommentTok{\#25\% danych}
\NormalTok{I\_t }\OtherTok{\textless{}{-}} \FunctionTok{setdiff}\NormalTok{(}\FunctionTok{setdiff}\NormalTok{(I, I\_l), I\_v) }\CommentTok{\#25\% danych}
\end{Highlighting}
\end{Shaded}

Przypiszmy im konkretne dane, dzieląc nasz wyjściowy zbiór danych na 3
podzbiory: próbę uczącą, walidacyjną oraz testową o udziale procentowym,
odpowiednio, 50\%, 25\% oraz 25\% danych wyjściowych.

\begin{Shaded}
\begin{Highlighting}[]
\NormalTok{lrn }\OtherTok{\textless{}{-}}\NormalTok{ white\_numeric[I\_l,] }\CommentTok{\#próba ucząca}
\NormalTok{val }\OtherTok{\textless{}{-}}\NormalTok{ white\_numeric[I\_v,] }\CommentTok{\#próba walidacyjna}
\NormalTok{tst }\OtherTok{\textless{}{-}}\NormalTok{ white\_numeric[I\_t,] }\CommentTok{\#próba testowa}
\end{Highlighting}
\end{Shaded}

W ten sposób podzieliliśmy zbiór danych na trzy podzbiory, które będą
nam pomocne do dalszej analizy.

\subsection{Korelacja zmiennych}\label{korelacja-zmiennych}

Zanim utworzymy model regresji liniowej, zobaczmy jak prezentuje się
wykres korelacji poszczególnych zmiennych w naszej próbie uczącej.

\begin{Shaded}
\begin{Highlighting}[]
\NormalTok{white\_matrix }\OtherTok{\textless{}{-}} \FunctionTok{cor}\NormalTok{(lrn)}
\FunctionTok{corrplot}\NormalTok{(white\_matrix, }\AttributeTok{type =} \StringTok{"lower"}\NormalTok{)}
\end{Highlighting}
\end{Shaded}

\includegraphics{Analiza-win_files/figure-latex/unnamed-chunk-13-1.pdf}
\emph{Korelacja} określa wzajemne powiązanie między wybranymi zmiennymi.
Wyrazem liczbowym korelacji jest współczynnik korelacji (R Pearsona)
zawierający się w przedziale {[}-1,1{]}.

Wartość dodatnia korelacji oznacza, że wraz ze wzrostem jednej cechy,
następuje wzrost drugiej. Wartość ujemna natomiast określa, że wraz ze
wzrostem jednej, następuje spadek drugiej cechy. W takim razie wartości
najbliżej krańców przedziału wskazują największe korelacje. Z drugiej
strony wartości bliżej zera wskazują na brak powiązania zmiennych.

Widzimy, że najbardziej skorelowaną zmienną ze zmienną `quality', jest
`alcohol' oraz `density', gdzie są to zmienne odpowiednio skorelowane
dodatnio oraz ujemnie.

Warto też zauważyć, że mocną wzajemną korelację wykazują zmienne
`density' i `residual.sugar' oraz zmienne `alcohol' i `density'. Mając
na uwadze te obserwacje, stworzymy modele bez `residual.sugar' i
`alcohol', żeby zobaczyć, jakie wyniki wtedy dostaniemy.

Zastosujemy również \emph{metodę głównych składowych (PCA)}, żeby
porównać oba podejścia.

\emph{Metoda głównych składowych (PCA)} polega na transformacji
oryginalnego zbioru zmiennych na nowy zestaw nieskorelowanych ze sobą
zmiennych, zwanych składowymi głównymi.

\section{Modele regresji liniowej}\label{modele-regresji-liniowej}

\subsection{Modele podstawowe}\label{modele-podstawowe}

Stworzymy modele regresji liniowej zmiennej `quality' względem,
odpowiednio:

\begin{itemize}
\tightlist
\item
  wszystkich zmiennych występujących w próbce uczącej (full),
\item
  wszystkich zmiennych występujących w próbce uczącej z wyrzuceniem
  zmiennej `residual.sugar' (no sugar),
\item
  wszystkich zmiennych występujących w próbce uczącej z wyrzuceniem
  zmiennej `alcohol' (no alcohol),
\item
  wszystkich zmiennych występujących w próbce uczącej z wyrzuceniem
  zmiennych `residual.sugar' oraz `alcohol' (no sugar, no alcohol).
\end{itemize}

Tworzymy w tym celu funkcję, żeby zautomatyzować ten proces.

\begin{Shaded}
\begin{Highlighting}[]
\NormalTok{models }\OtherTok{\textless{}\textless{}{-}} \FunctionTok{list}\NormalTok{()}
\NormalTok{countM  }\OtherTok{=} \DecValTok{0}
\NormalTok{add\_model }\OtherTok{=} \ControlFlowTok{function}\NormalTok{(name, model, ispca)\{}
\NormalTok{  models }\OtherTok{\textless{}\textless{}{-}} \FunctionTok{append}\NormalTok{(models, }\FunctionTok{list}\NormalTok{(name, model, ispca))}
\NormalTok{  countM }\OtherTok{\textless{}\textless{}{-}}\NormalTok{ countM }\SpecialCharTok{+} \DecValTok{1}
\NormalTok{\}}
\NormalTok{get\_name  }\OtherTok{=} \ControlFlowTok{function}\NormalTok{(index)\{models[[index}\SpecialCharTok{*}\DecValTok{3{-}3}\SpecialCharTok{+}\DecValTok{1}\NormalTok{]]\}}
\NormalTok{get\_model }\OtherTok{=} \ControlFlowTok{function}\NormalTok{(index)\{models[[index}\SpecialCharTok{*}\DecValTok{3{-}3}\SpecialCharTok{+}\DecValTok{2}\NormalTok{]]\}}
\NormalTok{is\_pca    }\OtherTok{=} \ControlFlowTok{function}\NormalTok{(index)\{models[[index}\SpecialCharTok{*}\DecValTok{3{-}3}\SpecialCharTok{+}\DecValTok{3}\NormalTok{]]\}}

\FunctionTok{add\_model}\NormalTok{(}\StringTok{\textquotesingle{}full\textquotesingle{}}\NormalTok{,                 }\FunctionTok{lm}\NormalTok{(quality }\SpecialCharTok{\textasciitilde{}}\NormalTok{ .,                            }\AttributeTok{data =}\NormalTok{ lrn), }\ConstantTok{FALSE}\NormalTok{)}
\FunctionTok{add\_model}\NormalTok{(}\StringTok{\textquotesingle{}no sugar\textquotesingle{}}\NormalTok{,             }\FunctionTok{lm}\NormalTok{(quality }\SpecialCharTok{\textasciitilde{}}\NormalTok{ . }\SpecialCharTok{{-}}\NormalTok{ residual.sugar,           }\AttributeTok{data =}\NormalTok{ lrn), }\ConstantTok{FALSE}\NormalTok{)}
\FunctionTok{add\_model}\NormalTok{(}\StringTok{\textquotesingle{}no alcohol\textquotesingle{}}\NormalTok{,           }\FunctionTok{lm}\NormalTok{(quality }\SpecialCharTok{\textasciitilde{}}\NormalTok{ .                  }\SpecialCharTok{{-}}\NormalTok{ alcohol, }\AttributeTok{data =}\NormalTok{ lrn), }\ConstantTok{FALSE}\NormalTok{)}
\FunctionTok{add\_model}\NormalTok{(}\StringTok{\textquotesingle{}no sugar, no alcohol\textquotesingle{}}\NormalTok{, }\FunctionTok{lm}\NormalTok{(quality }\SpecialCharTok{\textasciitilde{}}\NormalTok{ . }\SpecialCharTok{{-}}\NormalTok{ residual.sugar }\SpecialCharTok{{-}}\NormalTok{ alcohol, }\AttributeTok{data =}\NormalTok{ lrn), }\ConstantTok{FALSE}\NormalTok{)}
\end{Highlighting}
\end{Shaded}

Poniżej wyświetlamy podsumowanie każdego modelu w podanej wyżej
kolejności.

\begin{Shaded}
\begin{Highlighting}[]
\ControlFlowTok{for}\NormalTok{(i }\ControlFlowTok{in} \DecValTok{1}\SpecialCharTok{:}\NormalTok{countM)\{}
  \FunctionTok{print}\NormalTok{(}\FunctionTok{get\_name}\NormalTok{(i))}
  \FunctionTok{print}\NormalTok{(}\FunctionTok{summary}\NormalTok{(}\FunctionTok{get\_model}\NormalTok{(i)))}
\NormalTok{\}}
\end{Highlighting}
\end{Shaded}

\begin{verbatim}
## [1] "full"
## 
## Call:
## lm(formula = quality ~ ., data = lrn)
## 
## Residuals:
##     Min      1Q  Median      3Q     Max 
## -3.4424 -0.4903 -0.0343  0.4565  2.4520 
## 
## Coefficients:
##                        Estimate Std. Error t value Pr(>|t|)    
## (Intercept)           1.101e+02  2.304e+01   4.779 1.87e-06 ***
## fixed.acidity         3.955e-02  2.752e-02   1.437    0.151    
## volatile.acidity     -1.904e+00  1.576e-01 -12.085  < 2e-16 ***
## citric.acid           1.402e-01  1.302e-01   1.076    0.282    
## residual.sugar        7.067e-02  9.747e-03   7.251 5.56e-13 ***
## chlorides             1.601e-01  7.646e-01   0.209    0.834    
## free.sulfur.dioxide   4.733e-03  1.205e-03   3.928 8.79e-05 ***
## total.sulfur.dioxide -7.247e-04  5.193e-04  -1.396    0.163    
## density              -1.101e+02  2.339e+01  -4.707 2.65e-06 ***
## pH                    6.457e-01  1.424e-01   4.536 6.01e-06 ***
## sulphates             5.512e-01  1.378e-01   3.999 6.55e-05 ***
## alcohol               2.446e-01  3.013e-02   8.117 7.50e-16 ***
## ---
## Signif. codes:  0 '***' 0.001 '**' 0.01 '*' 0.05 '.' 0.1 ' ' 1
## 
## Residual standard error: 0.7422 on 2423 degrees of freedom
## Multiple R-squared:  0.2916, Adjusted R-squared:  0.2884 
## F-statistic: 90.68 on 11 and 2423 DF,  p-value: < 2.2e-16
## 
## [1] "no sugar"
## 
## Call:
## lm(formula = quality ~ . - residual.sugar, data = lrn)
## 
## Residuals:
##     Min      1Q  Median      3Q     Max 
## -3.3757 -0.5003 -0.0251  0.4502  2.4930 
## 
## Coefficients:
##                        Estimate Std. Error t value Pr(>|t|)    
## (Intercept)          -4.531e+01  8.537e+00  -5.307 1.21e-07 ***
## fixed.acidity        -8.727e-02  2.147e-02  -4.064 4.97e-05 ***
## volatile.acidity     -1.992e+00  1.588e-01 -12.549  < 2e-16 ***
## citric.acid           1.179e-01  1.316e-01   0.896   0.3704    
## chlorides            -1.040e+00  7.544e-01  -1.378   0.1683    
## free.sulfur.dioxide   6.298e-03  1.198e-03   5.258 1.59e-07 ***
## total.sulfur.dioxide -1.008e-03  5.233e-04  -1.926   0.0542 .  
## density               4.806e+01  8.533e+00   5.633 1.98e-08 ***
## pH                    2.352e-02  1.148e-01   0.205   0.8376    
## sulphates             3.055e-01  1.350e-01   2.263   0.0237 *  
## alcohol               4.056e-01  2.057e-02  19.716  < 2e-16 ***
## ---
## Signif. codes:  0 '***' 0.001 '**' 0.01 '*' 0.05 '.' 0.1 ' ' 1
## 
## Residual standard error: 0.7501 on 2424 degrees of freedom
## Multiple R-squared:  0.2762, Adjusted R-squared:  0.2733 
## F-statistic: 92.52 on 10 and 2424 DF,  p-value: < 2.2e-16
## 
## [1] "no alcohol"
## 
## Call:
## lm(formula = quality ~ . - alcohol, data = lrn)
## 
## Residuals:
##     Min      1Q  Median      3Q     Max 
## -3.4664 -0.4834 -0.0523  0.4578  5.3288 
## 
## Coefficients:
##                        Estimate Std. Error t value Pr(>|t|)    
## (Intercept)           2.710e+02  1.190e+01  22.773  < 2e-16 ***
## fixed.acidity         1.599e-01  2.349e-02   6.807 1.25e-11 ***
## volatile.acidity     -1.620e+00  1.557e-01 -10.408  < 2e-16 ***
## citric.acid           2.251e-01  1.316e-01   1.711  0.08718 .  
## residual.sugar        1.290e-01  6.673e-03  19.331  < 2e-16 ***
## chlorides             1.788e-02  7.746e-01   0.023  0.98158    
## free.sulfur.dioxide   3.853e-03  1.216e-03   3.168  0.00155 ** 
## total.sulfur.dioxide -8.445e-04  5.260e-04  -1.606  0.10850    
## density              -2.727e+02  1.224e+01 -22.275  < 2e-16 ***
## pH                    1.245e+00  1.233e-01  10.097  < 2e-16 ***
## sulphates             8.053e-01  1.360e-01   5.921 3.65e-09 ***
## ---
## Signif. codes:  0 '***' 0.001 '**' 0.01 '*' 0.05 '.' 0.1 ' ' 1
## 
## Residual standard error: 0.7521 on 2424 degrees of freedom
## Multiple R-squared:  0.2724, Adjusted R-squared:  0.2694 
## F-statistic: 90.73 on 10 and 2424 DF,  p-value: < 2.2e-16
## 
## [1] "no sugar, no alcohol"
## 
## Call:
## lm(formula = quality ~ . - residual.sugar - alcohol, data = lrn)
## 
## Residuals:
##     Min      1Q  Median      3Q     Max 
## -3.2628 -0.5514 -0.0178  0.4875  4.2115 
## 
## Coefficients:
##                        Estimate Std. Error t value Pr(>|t|)    
## (Intercept)           7.322e+01  6.528e+00  11.217  < 2e-16 ***
## fixed.acidity        -3.025e-02  2.292e-02  -1.320 0.186913    
## volatile.acidity     -1.225e+00  1.658e-01  -7.393 1.97e-13 ***
## citric.acid           3.597e-01  1.411e-01   2.549 0.010865 *  
## chlorides            -5.154e+00  7.808e-01  -6.601 4.99e-11 ***
## free.sulfur.dioxide   7.792e-03  1.288e-03   6.052 1.66e-09 ***
## total.sulfur.dioxide -2.293e-03  5.592e-04  -4.101 4.25e-05 ***
## density              -6.836e+01  6.634e+00 -10.305  < 2e-16 ***
## pH                    3.356e-01  1.224e-01   2.741 0.006175 ** 
## sulphates             4.922e-01  1.450e-01   3.393 0.000702 ***
## ---
## Signif. codes:  0 '***' 0.001 '**' 0.01 '*' 0.05 '.' 0.1 ' ' 1
## 
## Residual standard error: 0.8078 on 2425 degrees of freedom
## Multiple R-squared:  0.1602, Adjusted R-squared:  0.1571 
## F-statistic: 51.39 on 9 and 2425 DF,  p-value: < 2.2e-16
\end{verbatim}

Już na pierwszy rzut oka widać, że ostatni model nie jest w żaden sposób
konkurencyjny względem reszty modeli, biorąc pod uwagę chociażby
\emph{skorygowany współczynnik determinacji}, który jest
nieporównywalnie mniejszy w stosunku do reszty modeli. Jednak w celu
zweryfikowania, czy modele mniejsze są adekwatne, posłużymy się testem
ANOVA.

Jak chodzi o interpretację \emph{skorygowanego współczynnika
determinacji}, to przyjmuje się, że wartości bliżej zera oznacza model,
który nie ma wartości predykcyjnej.

\subsubsection{ANOVA}\label{anova}

\emph{ANOVA}, Analiza wariancji to rodzina modeli statystycznych i
powiązanych z nimi metod estymacji i wnioskowania wykorzystywanych do
analizy różnic pomiędzy średnimi w różnych populacjach, np. w zależności
od jednego lub wielu działających równoczeeśnie czynników. W
najprostszej formie ANOVA stanowi test statystyczny sprawdzający czy
dwie lub więcej średnich w populacjach jest sobie równych.

Nasz test statystyczny ma postać:

\begin{itemize}
\tightlist
\item
  H0: mniejszy model jest adekwatny,
\item
  H1: mniejszy model nie jest adekwatny.
\end{itemize}

\begin{Shaded}
\begin{Highlighting}[]
\ControlFlowTok{for}\NormalTok{(i }\ControlFlowTok{in} \DecValTok{2}\SpecialCharTok{:}\NormalTok{countM)\{}
  \FunctionTok{print}\NormalTok{(}\FunctionTok{paste}\NormalTok{(}\FunctionTok{get\_name}\NormalTok{(}\DecValTok{1}\NormalTok{), }\StringTok{\textquotesingle{}vs\textquotesingle{}}\NormalTok{, }\FunctionTok{get\_name}\NormalTok{(i), }\AttributeTok{sep=}\StringTok{\textquotesingle{} \textquotesingle{}}\NormalTok{))}
  \FunctionTok{print}\NormalTok{(}\FunctionTok{anova}\NormalTok{(}\FunctionTok{get\_model}\NormalTok{(}\DecValTok{1}\NormalTok{), }\FunctionTok{get\_model}\NormalTok{(i)))}
\NormalTok{\}}
\end{Highlighting}
\end{Shaded}

\begin{verbatim}
## [1] "full vs no sugar"
## Analysis of Variance Table
## 
## Model 1: quality ~ fixed.acidity + volatile.acidity + citric.acid + residual.sugar + 
##     chlorides + free.sulfur.dioxide + total.sulfur.dioxide + 
##     density + pH + sulphates + alcohol
## Model 2: quality ~ (fixed.acidity + volatile.acidity + citric.acid + residual.sugar + 
##     chlorides + free.sulfur.dioxide + total.sulfur.dioxide + 
##     density + pH + sulphates + alcohol) - residual.sugar
##   Res.Df    RSS Df Sum of Sq     F    Pr(>F)    
## 1   2423 1334.8                                 
## 2   2424 1363.7 -1   -28.959 52.57 5.559e-13 ***
## ---
## Signif. codes:  0 '***' 0.001 '**' 0.01 '*' 0.05 '.' 0.1 ' ' 1
## [1] "full vs no alcohol"
## Analysis of Variance Table
## 
## Model 1: quality ~ fixed.acidity + volatile.acidity + citric.acid + residual.sugar + 
##     chlorides + free.sulfur.dioxide + total.sulfur.dioxide + 
##     density + pH + sulphates + alcohol
## Model 2: quality ~ (fixed.acidity + volatile.acidity + citric.acid + residual.sugar + 
##     chlorides + free.sulfur.dioxide + total.sulfur.dioxide + 
##     density + pH + sulphates + alcohol) - alcohol
##   Res.Df    RSS Df Sum of Sq      F    Pr(>F)    
## 1   2423 1334.8                                  
## 2   2424 1371.0 -1   -36.297 65.891 7.501e-16 ***
## ---
## Signif. codes:  0 '***' 0.001 '**' 0.01 '*' 0.05 '.' 0.1 ' ' 1
## [1] "full vs no sugar, no alcohol"
## Analysis of Variance Table
## 
## Model 1: quality ~ fixed.acidity + volatile.acidity + citric.acid + residual.sugar + 
##     chlorides + free.sulfur.dioxide + total.sulfur.dioxide + 
##     density + pH + sulphates + alcohol
## Model 2: quality ~ (fixed.acidity + volatile.acidity + citric.acid + residual.sugar + 
##     chlorides + free.sulfur.dioxide + total.sulfur.dioxide + 
##     density + pH + sulphates + alcohol) - residual.sugar - alcohol
##   Res.Df    RSS Df Sum of Sq      F    Pr(>F)    
## 1   2423 1334.8                                  
## 2   2425 1582.4 -2   -247.66 224.79 < 2.2e-16 ***
## ---
## Signif. codes:  0 '***' 0.001 '**' 0.01 '*' 0.05 '.' 0.1 ' ' 1
\end{verbatim}

Widzimy stąd, że na poziomie istotności α=0.05 jesteśmy w stanie
odrzucić hipotezę H0. Oznacza to, że żaden z modeli powstałych przez
wyrzucenie zmiennych mocno skorelowanych z density nie jest adekwatny.

\subsubsection{Metoda wstecznej eliminacji i
ANOVA}\label{metoda-wstecznej-eliminacji-i-anova}

W tym podrozdziale zastosujemy \emph{metodę wstecznej eliminacji}.
Polega ona na kolejnym wyrzucaniu zmiennych, które są najmniej istotne w
modelu. Dzięki temu dostaniemy nowe modele. które posłużą nam do dalszej
analizy.

Zaznaczmy, że będziemy wyrzucać zmienne z pierwszego modelu, czyli z
modelu regresji liniowej zmiennej `quality' względem wszystkich
zmiennych występujących w próbce uczącej.

\begin{Shaded}
\begin{Highlighting}[]
\NormalTok{elim }\OtherTok{\textless{}{-}} \FunctionTok{add\_model}\NormalTok{(}\StringTok{\textquotesingle{}no acid\textquotesingle{}}\NormalTok{, }\FunctionTok{lm}\NormalTok{(quality }\SpecialCharTok{\textasciitilde{}}\NormalTok{ . }\SpecialCharTok{{-}}\NormalTok{ citric.acid, }\AttributeTok{data =}\NormalTok{ lrn), }\ConstantTok{FALSE}\NormalTok{)}
\FunctionTok{anova}\NormalTok{(}\FunctionTok{get\_model}\NormalTok{(}\DecValTok{1}\NormalTok{), }\FunctionTok{get\_model}\NormalTok{(countM))}
\end{Highlighting}
\end{Shaded}

\begin{verbatim}
## Analysis of Variance Table
## 
## Model 1: quality ~ fixed.acidity + volatile.acidity + citric.acid + residual.sugar + 
##     chlorides + free.sulfur.dioxide + total.sulfur.dioxide + 
##     density + pH + sulphates + alcohol
## Model 2: quality ~ (fixed.acidity + volatile.acidity + citric.acid + residual.sugar + 
##     chlorides + free.sulfur.dioxide + total.sulfur.dioxide + 
##     density + pH + sulphates + alcohol) - citric.acid
##   Res.Df    RSS Df Sum of Sq      F Pr(>F)
## 1   2423 1334.8                           
## 2   2424 1335.4 -1    -0.638 1.1582 0.2819
\end{verbatim}

\begin{Shaded}
\begin{Highlighting}[]
\FunctionTok{summary}\NormalTok{(}\FunctionTok{get\_model}\NormalTok{(countM))}
\end{Highlighting}
\end{Shaded}

\begin{verbatim}
## 
## Call:
## lm(formula = quality ~ . - citric.acid, data = lrn)
## 
## Residuals:
##     Min      1Q  Median      3Q     Max 
## -3.4668 -0.4857 -0.0355  0.4556  2.4624 
## 
## Coefficients:
##                        Estimate Std. Error t value Pr(>|t|)    
## (Intercept)           1.090e+02  2.302e+01   4.735 2.32e-06 ***
## fixed.acidity         4.359e-02  2.727e-02   1.599    0.110    
## volatile.acidity     -1.934e+00  1.551e-01 -12.470  < 2e-16 ***
## residual.sugar        7.042e-02  9.744e-03   7.227 6.59e-13 ***
## chlorides             2.955e-01  7.542e-01   0.392    0.695    
## free.sulfur.dioxide   4.771e-03  1.204e-03   3.962 7.66e-05 ***
## total.sulfur.dioxide -6.988e-04  5.187e-04  -1.347    0.178    
## density              -1.089e+02  2.337e+01  -4.662 3.30e-06 ***
## pH                    6.348e-01  1.420e-01   4.471 8.15e-06 ***
## sulphates             5.603e-01  1.376e-01   4.073 4.79e-05 ***
## alcohol               2.472e-01  3.003e-02   8.230 3.02e-16 ***
## ---
## Signif. codes:  0 '***' 0.001 '**' 0.01 '*' 0.05 '.' 0.1 ' ' 1
## 
## Residual standard error: 0.7422 on 2424 degrees of freedom
## Multiple R-squared:  0.2913, Adjusted R-squared:  0.2884 
## F-statistic: 99.63 on 10 and 2424 DF,  p-value: < 2.2e-16
\end{verbatim}

Przed wyświetleniem podsumowania zastosowaliśmy od razu test ANOVA, żeby
zweryfikować, czy mniejszy model jest adekwatny. Na poziomie istotności
α = 0,05 nie jesteśmy w stanie odrzucić hipotezy H0.

\begin{Shaded}
\begin{Highlighting}[]
\FunctionTok{add\_model}\NormalTok{(}\StringTok{\textquotesingle{}no acid, no chlorides\textquotesingle{}}\NormalTok{, }\FunctionTok{lm}\NormalTok{(quality }\SpecialCharTok{\textasciitilde{}}\NormalTok{ . }\SpecialCharTok{{-}}\NormalTok{ citric.acid }\SpecialCharTok{{-}}\NormalTok{ chlorides, }\AttributeTok{data =}\NormalTok{ lrn), }\ConstantTok{FALSE}\NormalTok{)}
\FunctionTok{anova}\NormalTok{(}\FunctionTok{get\_model}\NormalTok{(}\DecValTok{1}\NormalTok{), }\FunctionTok{get\_model}\NormalTok{(countM))}
\end{Highlighting}
\end{Shaded}

\begin{verbatim}
## Analysis of Variance Table
## 
## Model 1: quality ~ fixed.acidity + volatile.acidity + citric.acid + residual.sugar + 
##     chlorides + free.sulfur.dioxide + total.sulfur.dioxide + 
##     density + pH + sulphates + alcohol
## Model 2: quality ~ (fixed.acidity + volatile.acidity + citric.acid + residual.sugar + 
##     chlorides + free.sulfur.dioxide + total.sulfur.dioxide + 
##     density + pH + sulphates + alcohol) - citric.acid - chlorides
##   Res.Df    RSS Df Sum of Sq      F Pr(>F)
## 1   2423 1334.8                           
## 2   2425 1335.5 -2  -0.72257 0.6558 0.5191
\end{verbatim}

\begin{Shaded}
\begin{Highlighting}[]
\FunctionTok{summary}\NormalTok{(}\FunctionTok{get\_model}\NormalTok{(countM))}
\end{Highlighting}
\end{Shaded}

\begin{verbatim}
## 
## Call:
## lm(formula = quality ~ . - citric.acid - chlorides, data = lrn)
## 
## Residuals:
##     Min      1Q  Median      3Q     Max 
## -3.4676 -0.4855 -0.0338  0.4601  2.4582 
## 
## Coefficients:
##                        Estimate Std. Error t value Pr(>|t|)    
## (Intercept)           1.073e+02  2.262e+01   4.744 2.22e-06 ***
## fixed.acidity         4.182e-02  2.688e-02   1.556    0.120    
## volatile.acidity     -1.929e+00  1.546e-01 -12.480  < 2e-16 ***
## residual.sugar        6.957e-02  9.496e-03   7.326 3.22e-13 ***
## free.sulfur.dioxide   4.785e-03  1.204e-03   3.975 7.24e-05 ***
## total.sulfur.dioxide -6.942e-04  5.185e-04  -1.339    0.181    
## density              -1.072e+02  2.294e+01  -4.673 3.13e-06 ***
## pH                    6.243e-01  1.394e-01   4.479 7.86e-06 ***
## sulphates             5.580e-01  1.374e-01   4.061 5.05e-05 ***
## alcohol               2.471e-01  3.003e-02   8.228 3.07e-16 ***
## ---
## Signif. codes:  0 '***' 0.001 '**' 0.01 '*' 0.05 '.' 0.1 ' ' 1
## 
## Residual standard error: 0.7421 on 2425 degrees of freedom
## Multiple R-squared:  0.2912, Adjusted R-squared:  0.2886 
## F-statistic: 110.7 on 9 and 2425 DF,  p-value: < 2.2e-16
\end{verbatim}

\begin{Shaded}
\begin{Highlighting}[]
\FunctionTok{add\_model}\NormalTok{(}\StringTok{\textquotesingle{}no acid, no chlorides, no totalsulf\textquotesingle{}}\NormalTok{, }\FunctionTok{lm}\NormalTok{(quality }\SpecialCharTok{\textasciitilde{}}\NormalTok{ . }\SpecialCharTok{{-}}\NormalTok{ citric.acid }\SpecialCharTok{{-}}\NormalTok{ chlorides }\SpecialCharTok{{-}}\NormalTok{ total.sulfur.dioxide, }\AttributeTok{data =}\NormalTok{ lrn), }\ConstantTok{FALSE}\NormalTok{)}
\FunctionTok{anova}\NormalTok{(}\FunctionTok{get\_model}\NormalTok{(}\DecValTok{1}\NormalTok{), }\FunctionTok{get\_model}\NormalTok{(countM))}
\end{Highlighting}
\end{Shaded}

\begin{verbatim}
## Analysis of Variance Table
## 
## Model 1: quality ~ fixed.acidity + volatile.acidity + citric.acid + residual.sugar + 
##     chlorides + free.sulfur.dioxide + total.sulfur.dioxide + 
##     density + pH + sulphates + alcohol
## Model 2: quality ~ (fixed.acidity + volatile.acidity + citric.acid + residual.sugar + 
##     chlorides + free.sulfur.dioxide + total.sulfur.dioxide + 
##     density + pH + sulphates + alcohol) - citric.acid - chlorides - 
##     total.sulfur.dioxide
##   Res.Df    RSS Df Sum of Sq      F Pr(>F)
## 1   2423 1334.8                           
## 2   2426 1336.5 -3   -1.7097 1.0345 0.3761
\end{verbatim}

\begin{Shaded}
\begin{Highlighting}[]
\FunctionTok{summary}\NormalTok{(}\FunctionTok{get\_model}\NormalTok{(countM))}
\end{Highlighting}
\end{Shaded}

\begin{verbatim}
## 
## Call:
## lm(formula = quality ~ . - citric.acid - chlorides - total.sulfur.dioxide, 
##     data = lrn)
## 
## Residuals:
##     Min      1Q  Median      3Q     Max 
## -3.4384 -0.4894 -0.0324  0.4625  2.4664 
## 
## Coefficients:
##                       Estimate Std. Error t value Pr(>|t|)    
## (Intercept)          1.120e+02  2.236e+01   5.006 5.95e-07 ***
## fixed.acidity        4.112e-02  2.688e-02   1.530    0.126    
## volatile.acidity    -1.969e+00  1.518e-01 -12.973  < 2e-16 ***
## residual.sugar       7.063e-02  9.465e-03   7.463 1.17e-13 ***
## free.sulfur.dioxide  3.846e-03  9.785e-04   3.931 8.71e-05 ***
## density             -1.119e+02  2.268e+01  -4.935 8.56e-07 ***
## pH                   6.203e-01  1.394e-01   4.451 8.95e-06 ***
## sulphates            5.426e-01  1.370e-01   3.962 7.65e-05 ***
## alcohol              2.481e-01  3.002e-02   8.263 2.31e-16 ***
## ---
## Signif. codes:  0 '***' 0.001 '**' 0.01 '*' 0.05 '.' 0.1 ' ' 1
## 
## Residual standard error: 0.7422 on 2426 degrees of freedom
## Multiple R-squared:  0.2907, Adjusted R-squared:  0.2884 
## F-statistic: 124.3 on 8 and 2426 DF,  p-value: < 2.2e-16
\end{verbatim}

Podobna sytuacja występuje w dwóch kolejnych modelach. Nie jesteśmy w
stanie już dalej odrzucać zmiennych, ponieważ na poziomie istotności α =
0.05 kolejne zmienne są istotne, więc usuwanie ich nie jest sensowne.

\subsection{Analiza składowych
głównych}\label{analiza-skux142adowych-gux142uxf3wnych}

W celu zbadania innego podejścia do tworzenia modelów wykorzystamy
metodę składowych głównych.

\begin{Shaded}
\begin{Highlighting}[]
\FunctionTok{par}\NormalTok{(}\AttributeTok{mfrow =} \FunctionTok{c}\NormalTok{(}\DecValTok{1}\NormalTok{, }\DecValTok{2}\NormalTok{))}

\NormalTok{pca     }\OtherTok{\textless{}{-}} \FunctionTok{prcomp}\NormalTok{(lrn[}\DecValTok{1}\SpecialCharTok{:}\DecValTok{11}\NormalTok{], }\AttributeTok{scale. =} \ConstantTok{TRUE}\NormalTok{)}
\NormalTok{pca\_var }\OtherTok{\textless{}{-}}\NormalTok{ pca}\SpecialCharTok{$}\NormalTok{sdev}\SpecialCharTok{\^{}}\DecValTok{2}
\FunctionTok{plot}\NormalTok{(pca\_var }\SpecialCharTok{/} \FunctionTok{sum}\NormalTok{(pca\_var), }\AttributeTok{xlab =} \StringTok{"Składowe główne"}\NormalTok{,}
 \AttributeTok{ylab =} \StringTok{"Proporcja wariancji objaśnianej"}\NormalTok{,}
 \AttributeTok{type =} \StringTok{"b"}\NormalTok{)}

\NormalTok{pca\_data }\OtherTok{\textless{}{-}} \FunctionTok{data.frame}\NormalTok{(}\AttributeTok{quality =}\NormalTok{ lrn}\SpecialCharTok{$}\NormalTok{quality, pca}\SpecialCharTok{$}\NormalTok{x)}

\FunctionTok{corrplot}\NormalTok{(}\FunctionTok{cor}\NormalTok{(pca\_data), }\AttributeTok{type =} \StringTok{"lower"}\NormalTok{)}
\end{Highlighting}
\end{Shaded}

\includegraphics{Analiza-win_files/figure-latex/unnamed-chunk-23-1.pdf}

Jak widać na wykresie dwie zmienne, które mają najmniejsze znaczenie są
poniżej poziomu istotności α = 0,05. Z wykresu obok odczytujemy także,
że udało nam się utworzyć zbiór danych, w którym wszystkie zmienne (poza
`quality') są niezależne. Zastosujemy także tutaj metodę wstecznej
eliminacji, żeby pozbyć się zmiennych, które są nieistotne w tym modelu.

\subsection{Metoda wstecznej eliminacji i
ANOVA}\label{metoda-wstecznej-eliminacji-i-anova-1}

Na początek tworzymy model PCA, z którego będziemy wyrzucać zmienne oraz
od razu sprawdzimy, czy mniejsze modele są adekwatne względem tego
modelu.

\begin{Shaded}
\begin{Highlighting}[]
\NormalTok{pca\_base }\OtherTok{\textless{}{-}} \FunctionTok{add\_model}\NormalTok{(}\StringTok{\textquotesingle{}pca\textquotesingle{}}\NormalTok{, }\FunctionTok{lm}\NormalTok{(quality }\SpecialCharTok{\textasciitilde{}}\NormalTok{ ., }\AttributeTok{data =}\NormalTok{ pca\_data), }\ConstantTok{TRUE}\NormalTok{)}
\FunctionTok{summary}\NormalTok{(}\FunctionTok{get\_model}\NormalTok{(countM))}
\end{Highlighting}
\end{Shaded}

\begin{verbatim}
## 
## Call:
## lm(formula = quality ~ ., data = pca_data)
## 
## Residuals:
##     Min      1Q  Median      3Q     Max 
## -3.4424 -0.4903 -0.0343  0.4565  2.4520 
## 
## Coefficients:
##              Estimate Std. Error t value Pr(>|t|)    
## (Intercept)  5.880493   0.015041 390.967  < 2e-16 ***
## PC1          0.140118   0.008396  16.689  < 2e-16 ***
## PC2          0.043043   0.012142   3.545  0.00040 ***
## PC3         -0.163705   0.013533 -12.096  < 2e-16 ***
## PC4          0.181727   0.014750  12.320  < 2e-16 ***
## PC5          0.031171   0.015243   2.045  0.04097 *  
## PC6         -0.028545   0.015576  -1.833  0.06698 .  
## PC7         -0.098154   0.017661  -5.558 3.03e-08 ***
## PC8          0.184092   0.019160   9.608  < 2e-16 ***
## PC9          0.369938   0.023511  15.735  < 2e-16 ***
## PC10         0.112392   0.027739   4.052 5.24e-05 ***
## PC11         0.345491   0.093940   3.678  0.00024 ***
## ---
## Signif. codes:  0 '***' 0.001 '**' 0.01 '*' 0.05 '.' 0.1 ' ' 1
## 
## Residual standard error: 0.7422 on 2423 degrees of freedom
## Multiple R-squared:  0.2916, Adjusted R-squared:  0.2884 
## F-statistic: 90.68 on 11 and 2423 DF,  p-value: < 2.2e-16
\end{verbatim}

\begin{Shaded}
\begin{Highlighting}[]
\FunctionTok{add\_model}\NormalTok{(}\StringTok{\textquotesingle{}pca no pc6\textquotesingle{}}\NormalTok{, }\FunctionTok{lm}\NormalTok{(quality }\SpecialCharTok{\textasciitilde{}}\NormalTok{ . }\SpecialCharTok{{-}}\NormalTok{ PC6, }\AttributeTok{data =}\NormalTok{ pca\_data), }\ConstantTok{TRUE}\NormalTok{)}
\FunctionTok{anova}\NormalTok{(}\FunctionTok{get\_model}\NormalTok{(pca\_base), }\FunctionTok{get\_model}\NormalTok{(countM))}
\end{Highlighting}
\end{Shaded}

\begin{verbatim}
## Analysis of Variance Table
## 
## Model 1: quality ~ PC1 + PC2 + PC3 + PC4 + PC5 + PC6 + PC7 + PC8 + PC9 + 
##     PC10 + PC11
## Model 2: quality ~ (PC1 + PC2 + PC3 + PC4 + PC5 + PC6 + PC7 + PC8 + PC9 + 
##     PC10 + PC11) - PC6
##   Res.Df    RSS Df Sum of Sq      F  Pr(>F)  
## 1   2423 1334.8                              
## 2   2424 1336.6 -1   -1.8501 3.3585 0.06698 .
## ---
## Signif. codes:  0 '***' 0.001 '**' 0.01 '*' 0.05 '.' 0.1 ' ' 1
\end{verbatim}

\begin{Shaded}
\begin{Highlighting}[]
\FunctionTok{summary}\NormalTok{(}\FunctionTok{get\_model}\NormalTok{(countM))}
\end{Highlighting}
\end{Shaded}

\begin{verbatim}
## 
## Call:
## lm(formula = quality ~ . - PC6, data = pca_data)
## 
## Residuals:
##     Min      1Q  Median      3Q     Max 
## -3.4041 -0.4857 -0.0407  0.4605  2.4551 
## 
## Coefficients:
##             Estimate Std. Error t value Pr(>|t|)    
## (Intercept)  5.88049    0.01505 390.777  < 2e-16 ***
## PC1          0.14012    0.00840  16.681  < 2e-16 ***
## PC2          0.04304    0.01215   3.543 0.000403 ***
## PC3         -0.16370    0.01354 -12.090  < 2e-16 ***
## PC4          0.18173    0.01476  12.314  < 2e-16 ***
## PC5          0.03117    0.01525   2.044 0.041065 *  
## PC7         -0.09815    0.01767  -5.555 3.08e-08 ***
## PC8          0.18409    0.01917   9.603  < 2e-16 ***
## PC9          0.36994    0.02352  15.727  < 2e-16 ***
## PC10         0.11239    0.02775   4.050 5.29e-05 ***
## PC11         0.34549    0.09399   3.676 0.000242 ***
## ---
## Signif. codes:  0 '***' 0.001 '**' 0.01 '*' 0.05 '.' 0.1 ' ' 1
## 
## Residual standard error: 0.7426 on 2424 degrees of freedom
## Multiple R-squared:  0.2906, Adjusted R-squared:  0.2877 
## F-statistic: 99.31 on 10 and 2424 DF,  p-value: < 2.2e-16
\end{verbatim}

Jak widać model z wyrzuconym składnikiem `PC6' jest jak najbardziej
adekwatny. Spróbujmy wyrzucić jeszcze jeden składnik, tym razem `PC7'.

\begin{Shaded}
\begin{Highlighting}[]
\FunctionTok{add\_model}\NormalTok{(}\StringTok{\textquotesingle{}pca no pc6, no pc7\textquotesingle{}}\NormalTok{, }\FunctionTok{lm}\NormalTok{(quality }\SpecialCharTok{\textasciitilde{}}\NormalTok{ . }\SpecialCharTok{{-}}\NormalTok{ PC6 }\SpecialCharTok{{-}}\NormalTok{ PC7, }\AttributeTok{data =}\NormalTok{ pca\_data), }\ConstantTok{TRUE}\NormalTok{)}
\FunctionTok{anova}\NormalTok{(}\FunctionTok{get\_model}\NormalTok{(pca\_base), }\FunctionTok{get\_model}\NormalTok{(countM))}
\end{Highlighting}
\end{Shaded}

\begin{verbatim}
## Analysis of Variance Table
## 
## Model 1: quality ~ PC1 + PC2 + PC3 + PC4 + PC5 + PC6 + PC7 + PC8 + PC9 + 
##     PC10 + PC11
## Model 2: quality ~ (PC1 + PC2 + PC3 + PC4 + PC5 + PC6 + PC7 + PC8 + PC9 + 
##     PC10 + PC11) - PC6 - PC7
##   Res.Df    RSS Df Sum of Sq      F    Pr(>F)    
## 1   2423 1334.8                                  
## 2   2425 1353.6 -2   -18.865 17.123 4.127e-08 ***
## ---
## Signif. codes:  0 '***' 0.001 '**' 0.01 '*' 0.05 '.' 0.1 ' ' 1
\end{verbatim}

\begin{Shaded}
\begin{Highlighting}[]
\FunctionTok{summary}\NormalTok{(}\FunctionTok{get\_model}\NormalTok{(countM))}
\end{Highlighting}
\end{Shaded}

\begin{verbatim}
## 
## Call:
## lm(formula = quality ~ . - PC6 - PC7, data = pca_data)
## 
## Residuals:
##     Min      1Q  Median      3Q     Max 
## -3.5416 -0.4903 -0.0346  0.4557  2.4427 
## 
## Coefficients:
##              Estimate Std. Error t value Pr(>|t|)    
## (Intercept)  5.880493   0.015141 388.393  < 2e-16 ***
## PC1          0.140118   0.008451  16.579  < 2e-16 ***
## PC2          0.043043   0.012223   3.521 0.000437 ***
## PC3         -0.163705   0.013623 -12.017  < 2e-16 ***
## PC4          0.181727   0.014848  12.239  < 2e-16 ***
## PC5          0.031171   0.015344   2.031 0.042314 *  
## PC8          0.184092   0.019287   9.545  < 2e-16 ***
## PC9          0.369938   0.023667  15.631  < 2e-16 ***
## PC10         0.112392   0.027923   4.025 5.87e-05 ***
## PC11         0.345491   0.094563   3.654 0.000264 ***
## ---
## Signif. codes:  0 '***' 0.001 '**' 0.01 '*' 0.05 '.' 0.1 ' ' 1
## 
## Residual standard error: 0.7471 on 2425 degrees of freedom
## Multiple R-squared:  0.2816, Adjusted R-squared:  0.2789 
## F-statistic: 105.6 on 9 and 2425 DF,  p-value: < 2.2e-16
\end{verbatim}

Wynika stąd, że ten model także jest adekwatny, a przynajmniej wiemy, że
nie jesteśmy w stanie odrzucić hipotezy H0 na poziomie istotności α =
0,05. Dalsze wyrzucanie zmiennych nie ma sensu, ponieważ reszta
zmiennych jest jak najbardziej istotna.

\subsection{Diagnostyka}\label{diagnostyka}

Przyjdziemy teraz do diagnostyki naszych nowo utworzonych modeli.
Przyjrzymy się \emph{statystyce Cooka (odległość Cooka)}, żeby wykryć
obserwacje odstające, pozbyć się ich i stąd dostaniemy nowe modele.
Dodatkowo policzymy procentowy udział obserwacji wpływowych. Spojrzymy
też na wykresy resztowe naszych modeli i spróbujemy wyciągnąć wnioski.

Krótkie wyjaśnienie \emph{odległość Cooka} to miara wykorzystywana w
analizie regresji, służąca do wykrywania obserwacji odstających i
oszacowania ich wpływu na cały model statystyczny.

\subsubsection{Obserwacje odstające}\label{obserwacje-odstajux105ce}

Tworzymy pomocniczą funkcję, żeby maksymalnie zautomatyzować tworzenie
odległości Cooka oraz powstawanie wykresów. Pamiętamy, że za obserwację
wpływową w sensie odległości Cooka uchodzą obserwacje, dla których
odległość jest nie mniejsza od 1.

\begin{Shaded}
\begin{Highlighting}[]
\NormalTok{cook\_base }\OtherTok{=}\NormalTok{ countM}

\NormalTok{cook\_statistics }\OtherTok{=} \ControlFlowTok{function}\NormalTok{(name, model, ispca)\{}
\NormalTok{  cook }\OtherTok{\textless{}{-}} \FunctionTok{cooks.distance}\NormalTok{(model)}
  \FunctionTok{plot}\NormalTok{(cook, }\AttributeTok{xlab =} \StringTok{"Indeksy"}\NormalTok{,  }\AttributeTok{ylab =} \FunctionTok{paste}\NormalTok{(}\StringTok{"Odległości("}\NormalTok{, name, }\StringTok{")"}\NormalTok{))}
  \ControlFlowTok{if}\NormalTok{(}\FunctionTok{max}\NormalTok{(cook) }\SpecialCharTok{\textgreater{}=} \DecValTok{1}\NormalTok{)\{}
\NormalTok{    name }\OtherTok{\textless{}{-}} \FunctionTok{paste}\NormalTok{(}\StringTok{\textquotesingle{}cook\textquotesingle{}}\NormalTok{, name, }\AttributeTok{sep =} \StringTok{\textquotesingle{} \textquotesingle{}}\NormalTok{)}
    \ControlFlowTok{while}\NormalTok{(}\FunctionTok{max}\NormalTok{(cook) }\SpecialCharTok{\textgreater{}=} \DecValTok{1}\NormalTok{)\{}
\NormalTok{      model }\OtherTok{\textless{}{-}} \FunctionTok{update}\NormalTok{(model, }\AttributeTok{subset =}\NormalTok{ (cook }\SpecialCharTok{\textless{}} \FunctionTok{max}\NormalTok{(cook)))}
\NormalTok{      cook  }\OtherTok{\textless{}{-}} \FunctionTok{cooks.distance}\NormalTok{(model)}
\NormalTok{    \}}
    \FunctionTok{add\_model}\NormalTok{(name, model, ispca)}
\NormalTok{  \}}
\NormalTok{\}}

\FunctionTok{layout}\NormalTok{(}\FunctionTok{matrix}\NormalTok{(}\FunctionTok{c}\NormalTok{(}\DecValTok{1}\NormalTok{,}\DecValTok{2}\NormalTok{,}\DecValTok{3}\NormalTok{,}\DecValTok{4}\NormalTok{,}\DecValTok{5}\NormalTok{,}\DecValTok{6}\NormalTok{,}\DecValTok{7}\NormalTok{,}\DecValTok{8}\NormalTok{,}\DecValTok{9}\NormalTok{,}\DecValTok{10}\NormalTok{), }\DecValTok{2}\NormalTok{, }\DecValTok{5}\NormalTok{, }\AttributeTok{byrow =} \ConstantTok{TRUE}\NormalTok{))}

\ControlFlowTok{for}\NormalTok{(i }\ControlFlowTok{in} \DecValTok{1}\SpecialCharTok{:}\NormalTok{cook\_base)\{}
\NormalTok{  name  }\OtherTok{\textless{}{-}} \FunctionTok{get\_name}\NormalTok{(i)}
\NormalTok{  model }\OtherTok{\textless{}{-}} \FunctionTok{get\_model}\NormalTok{(i)}
  \FunctionTok{cook\_statistics}\NormalTok{(name, model, }\FunctionTok{is\_pca}\NormalTok{(i))}
\NormalTok{\}}

\FunctionTok{mtext}\NormalTok{(}\StringTok{"Odległość Cooka"}\NormalTok{, }\AttributeTok{side=}\DecValTok{3}\NormalTok{, }\AttributeTok{outer=}\ConstantTok{TRUE}\NormalTok{, }\AttributeTok{line=}\SpecialCharTok{{-}}\DecValTok{3}\NormalTok{)}
\end{Highlighting}
\end{Shaded}

\includegraphics{Analiza-win_files/figure-latex/unnamed-chunk-29-1.pdf}

Z powyższych wykresów jesteśmy w stanie wywnioskować, że modele `no
sugar', `no alcohol' oraz `no sugar, no alcohol' nie mają żadnych
odstających obserwacji, ponieważ odległość Cooka jest mniejsza niż 1.
Pozostałe modele posiadają obserwacje odstające, dlatego za pomocą
funkcji, która została stworzona na początku tego podrozdziału,
automatycznie stworzone zostały nowe modele z wyrzuceniem obserwacji
odstającej. Warto dodać, że żaden z modeli nie posiadał więcej niż
jednej obserwacji odstającej.

\subsubsection{Obserwacje wpływowe}\label{obserwacje-wpux142ywowe}

Zajmiemy się teraz obserwacjami wpływowymi. Wyświetlimy wykresy każdego
modelu wraz z linią pokazującą, od jakiego poziomu obserwacje są
obserwacjami wpływowymi. W tym celu robimy funkcję, żeby po raz kolejny
zautomatyzować tworzenie wykresów, ponieważ na chwilę obecną mamy ich
już 17.

\begin{Shaded}
\begin{Highlighting}[]
\FunctionTok{par}\NormalTok{(}\AttributeTok{mfrow =} \FunctionTok{c}\NormalTok{(}\DecValTok{9}\NormalTok{, }\DecValTok{3}\NormalTok{))}

\NormalTok{leverages }\OtherTok{\textless{}{-}} \FunctionTok{mapply}\NormalTok{(}\ControlFlowTok{function}\NormalTok{(index)\{}
\NormalTok{    name  }\OtherTok{\textless{}{-}} \FunctionTok{get\_name}\NormalTok{(index)}
\NormalTok{    model }\OtherTok{\textless{}{-}} \FunctionTok{get\_model}\NormalTok{(index)}
\NormalTok{    lev   }\OtherTok{\textless{}{-}} \FunctionTok{hat}\NormalTok{(}\FunctionTok{model.matrix}\NormalTok{(model))}
\NormalTok{    p }\OtherTok{\textless{}{-}}\NormalTok{ (index }\SpecialCharTok{{-}} \DecValTok{1}\NormalTok{) }\SpecialCharTok{\%\%} \DecValTok{3}
\NormalTok{    q }\OtherTok{\textless{}{-}} \FunctionTok{ifelse}\NormalTok{((index }\SpecialCharTok{{-}} \DecValTok{1}\NormalTok{) }\SpecialCharTok{\%\%} \DecValTok{6} \SpecialCharTok{\textgreater{}=} \DecValTok{3}\NormalTok{, }\DecValTok{0}\NormalTok{, }\DecValTok{1}\NormalTok{)}
    \FunctionTok{par}\NormalTok{(}\AttributeTok{fig =} \FunctionTok{c}\NormalTok{(}\DecValTok{1}\SpecialCharTok{/}\DecValTok{3} \SpecialCharTok{*}\NormalTok{ p,}\DecValTok{1}\SpecialCharTok{/}\DecValTok{3} \SpecialCharTok{+} \DecValTok{1}\SpecialCharTok{/}\DecValTok{3} \SpecialCharTok{*}\NormalTok{ p, }\FloatTok{0.5} \SpecialCharTok{*}\NormalTok{ q, }\FloatTok{0.5} \SpecialCharTok{+} \FloatTok{0.5} \SpecialCharTok{*}\NormalTok{ q), }\AttributeTok{new =}\NormalTok{ (index }\SpecialCharTok{\%\%} \DecValTok{6} \SpecialCharTok{!=} \DecValTok{1}\NormalTok{))}
    \FunctionTok{plot}\NormalTok{(lev, }\AttributeTok{xlab =} \StringTok{"Indeksy"}\NormalTok{, }\AttributeTok{ylab =} \StringTok{"Obserwacje wpływowe"}\NormalTok{, }\AttributeTok{main =}\NormalTok{ name)}
    \FunctionTok{abline}\NormalTok{(}\AttributeTok{h =} \DecValTok{2} \SpecialCharTok{*} \FunctionTok{sum}\NormalTok{(lev) }\SpecialCharTok{/} \FunctionTok{nrow}\NormalTok{(model}\SpecialCharTok{$}\NormalTok{model), }\AttributeTok{col =} \StringTok{\textquotesingle{}red\textquotesingle{}}\NormalTok{)}
\NormalTok{    lev}
\NormalTok{  \}, }\DecValTok{1}\SpecialCharTok{:}\NormalTok{countM)}
\end{Highlighting}
\end{Shaded}

\includegraphics{Analiza-win_files/figure-latex/unnamed-chunk-30-1.pdf}
\includegraphics{Analiza-win_files/figure-latex/unnamed-chunk-30-2.pdf}
\includegraphics{Analiza-win_files/figure-latex/unnamed-chunk-30-3.pdf}

Widać, że dopiero po odrzuceniu obserwacji odstających tak powstałe
modele mają bardziej przejrzyste wykresy. Jednak z nich nie jesteśmy w
stanie za wiele odczytać, dlatego policzymy procentowy udział obserwacji
wływowych i wyświetlimy wyniki w postaci przejrzystej tabelki.

\begin{Shaded}
\begin{Highlighting}[]
\FunctionTok{data.frame}\NormalTok{(}\AttributeTok{names =} \FunctionTok{mapply}\NormalTok{(get\_name, }\DecValTok{1}\SpecialCharTok{:}\NormalTok{countM), }
           \AttributeTok{percentages =} \FunctionTok{mapply}\NormalTok{(}\ControlFlowTok{function}\NormalTok{(index)\{ }
\NormalTok{                           model }\OtherTok{\textless{}{-}} \FunctionTok{get\_model}\NormalTok{(index)}
\NormalTok{                           lev   }\OtherTok{\textless{}{-}}\NormalTok{ leverages[[index]]}
                           \FunctionTok{paste}\NormalTok{(}\FunctionTok{round}\NormalTok{(}\FunctionTok{sum}\NormalTok{( }\FunctionTok{ifelse}\NormalTok{(lev }\SpecialCharTok{\textgreater{}} \DecValTok{2} \SpecialCharTok{*} \FunctionTok{sum}\NormalTok{(lev) }\SpecialCharTok{/} \FunctionTok{nrow}\NormalTok{(model}\SpecialCharTok{$}\NormalTok{model), }\DecValTok{1}\NormalTok{, }\DecValTok{0}\NormalTok{)) }\SpecialCharTok{/} \FunctionTok{nrow}\NormalTok{(model}\SpecialCharTok{$}\NormalTok{model) }\SpecialCharTok{*} \DecValTok{100}\NormalTok{, }\DecValTok{2}\NormalTok{), }\StringTok{\textquotesingle{}\%\textquotesingle{}}\NormalTok{)}
\NormalTok{                         \}, }\DecValTok{1}\SpecialCharTok{:}\NormalTok{countM))}
\end{Highlighting}
\end{Shaded}

\begin{verbatim}
##                                       names percentages
## 1                                      full      5.26 %
## 2                                  no sugar      6.12 %
## 3                                no alcohol      5.91 %
## 4                      no sugar, no alcohol      6.41 %
## 5                                   no acid      5.42 %
## 6                     no acid, no chlorides      4.64 %
## 7       no acid, no chlorides, no totalsulf      4.89 %
## 8                                       pca      5.26 %
## 9                                pca no pc6      6.04 %
## 10                       pca no pc6, no pc7      5.75 %
## 11                          cook no alcohol      6.08 %
## 12                             cook no acid      5.92 %
## 13               cook no acid, no chlorides      5.42 %
## 14 cook no acid, no chlorides, no totalsulf      5.55 %
## 15                  cook pca no pc6, no pc7      6.45 %
\end{verbatim}

Widzimy różne rozłożenie procentowe obserwacji wpływowych dla różnych
modeli.

\subsubsection{Wykresy resztowe}\label{wykresy-resztowe}

Spójrzmy jeszcze tylko na wykresy resztowe wszystkich modeli.

\begin{Shaded}
\begin{Highlighting}[]
\FunctionTok{par}\NormalTok{(}\AttributeTok{mfrow =} \FunctionTok{c}\NormalTok{(}\FunctionTok{ceiling}\NormalTok{(countM }\SpecialCharTok{/} \DecValTok{4}\NormalTok{), }\DecValTok{4}\NormalTok{))}

\ControlFlowTok{for}\NormalTok{(i }\ControlFlowTok{in} \DecValTok{1}\SpecialCharTok{:}\NormalTok{countM)\{}
\NormalTok{  model }\OtherTok{\textless{}{-}} \FunctionTok{get\_model}\NormalTok{(i)}
\NormalTok{  p }\OtherTok{\textless{}{-}}\NormalTok{ (i }\SpecialCharTok{{-}} \DecValTok{1}\NormalTok{) }\SpecialCharTok{\%\%} \DecValTok{4}
\NormalTok{  q }\OtherTok{\textless{}{-}} \FunctionTok{ifelse}\NormalTok{((i }\SpecialCharTok{{-}} \DecValTok{1}\NormalTok{) }\SpecialCharTok{\%\%} \DecValTok{8} \SpecialCharTok{\textgreater{}=} \DecValTok{4}\NormalTok{, }\DecValTok{0}\NormalTok{, }\DecValTok{1}\NormalTok{)}
  \FunctionTok{par}\NormalTok{(}\AttributeTok{fig =} \FunctionTok{c}\NormalTok{(}\DecValTok{1}\SpecialCharTok{/}\DecValTok{4} \SpecialCharTok{*}\NormalTok{ p, }\DecValTok{1}\SpecialCharTok{/}\DecValTok{4} \SpecialCharTok{+} \DecValTok{1}\SpecialCharTok{/}\DecValTok{4} \SpecialCharTok{*}\NormalTok{ p, }\FloatTok{0.5} \SpecialCharTok{*}\NormalTok{ q, }\FloatTok{0.5} \SpecialCharTok{+} \FloatTok{0.5} \SpecialCharTok{*}\NormalTok{ q), }\AttributeTok{new =}\NormalTok{ (i }\SpecialCharTok{\%\%} \DecValTok{8} \SpecialCharTok{!=} \DecValTok{1}\NormalTok{))}
  \FunctionTok{plot}\NormalTok{(model}\SpecialCharTok{$}\NormalTok{fit, model}\SpecialCharTok{$}\NormalTok{res, }\AttributeTok{xlab=}\StringTok{"Dopasowane"}\NormalTok{, }\AttributeTok{ylab=}\StringTok{"Reszty"}\NormalTok{, }\AttributeTok{main =} \FunctionTok{get\_name}\NormalTok{(i))}
  \FunctionTok{abline}\NormalTok{(}\AttributeTok{h =} \DecValTok{0}\NormalTok{, }\AttributeTok{col =} \StringTok{\textquotesingle{}red\textquotesingle{}}\NormalTok{)\}}
\end{Highlighting}
\end{Shaded}

\includegraphics{Analiza-win_files/figure-latex/unnamed-chunk-32-1.pdf}
\includegraphics{Analiza-win_files/figure-latex/unnamed-chunk-32-2.pdf}

W każdym z modeli występuje pewne odchylanie reszt od wartości
dopasowanych. Widzimy, że wykresy są bardzo podobne do siebie, a
wszystkie bardziej odstające punkty zostały usunięte podczas tworzenia
modeli stworzonych dzięki statystyce Cooka, co widać porównując
odpowiednio modele podstawowe ze zmodyfikowanymi. Jedynie wykresy modeli
`no alcohol' oraz `no sugar, no alcohol' znacząco różnią się od reszty
wykresów.

\subsection{Poprawność założeń modeli regresji
liniowej}\label{poprawnoux15bux107-zaux142oux17ceux144-modeli-regresji-liniowej}

\subsubsection{Normalność reszt}\label{normalnoux15bux107-reszt}

Przyjrzyjmy się najpierw wykresom.

\begin{Shaded}
\begin{Highlighting}[]
\FunctionTok{par}\NormalTok{(}\AttributeTok{mfrow =} \FunctionTok{c}\NormalTok{(}\FunctionTok{ceiling}\NormalTok{(countM }\SpecialCharTok{/} \DecValTok{4}\NormalTok{), }\DecValTok{4}\NormalTok{))}

\ControlFlowTok{for}\NormalTok{(i }\ControlFlowTok{in} \DecValTok{1}\SpecialCharTok{:}\NormalTok{countM)\{}
\NormalTok{  model }\OtherTok{\textless{}{-}} \FunctionTok{get\_model}\NormalTok{(i)}
\NormalTok{  p }\OtherTok{\textless{}{-}}\NormalTok{ (i }\SpecialCharTok{{-}} \DecValTok{1}\NormalTok{) }\SpecialCharTok{\%\%} \DecValTok{4}
\NormalTok{  q }\OtherTok{\textless{}{-}} \FunctionTok{ifelse}\NormalTok{((i }\SpecialCharTok{{-}} \DecValTok{1}\NormalTok{) }\SpecialCharTok{\%\%} \DecValTok{8} \SpecialCharTok{\textgreater{}=} \DecValTok{4}\NormalTok{, }\DecValTok{0}\NormalTok{, }\DecValTok{1}\NormalTok{)}
  \FunctionTok{par}\NormalTok{(}\AttributeTok{fig =} \FunctionTok{c}\NormalTok{(}\DecValTok{1}\SpecialCharTok{/}\DecValTok{4} \SpecialCharTok{*}\NormalTok{ p, }\DecValTok{1}\SpecialCharTok{/}\DecValTok{4} \SpecialCharTok{+} \DecValTok{1}\SpecialCharTok{/}\DecValTok{4} \SpecialCharTok{*}\NormalTok{ p, }\FloatTok{0.5} \SpecialCharTok{*}\NormalTok{ q, }\FloatTok{0.5} \SpecialCharTok{+} \FloatTok{0.5} \SpecialCharTok{*}\NormalTok{ q), }\AttributeTok{new =}\NormalTok{ (i }\SpecialCharTok{\%\%} \DecValTok{8} \SpecialCharTok{!=} \DecValTok{1}\NormalTok{))}
  \FunctionTok{qqnorm}\NormalTok{(}\FunctionTok{rstudent}\NormalTok{(model), }\AttributeTok{xlab =} \StringTok{"Teoretyczne Kwantyle"}\NormalTok{, }\AttributeTok{ylab =} \StringTok{"Studentyzowane reszty"}\NormalTok{, }\AttributeTok{main =} \FunctionTok{get\_name}\NormalTok{(i))}
  \FunctionTok{abline}\NormalTok{(}\DecValTok{0}\NormalTok{,}\DecValTok{1}\NormalTok{)}
\NormalTok{\}}
\end{Highlighting}
\end{Shaded}

\includegraphics{Analiza-win_files/figure-latex/unnamed-chunk-33-1.pdf}
\includegraphics{Analiza-win_files/figure-latex/unnamed-chunk-33-2.pdf}

Z wykresów widzimy, że rozkład zmiennych resztowych nie jest zupełnie
normalny, ale jest to łagodne odstępstwo od założenia normalności,
ponadto próbka jest duża, więc może być zignorowane. Niemniej
przeprowadźmy jeszcze test statystyczny Shapiro--Wilk.

Ten test ma następujące hipotezy:

H0: reszty mają rozkład normalny,

H1: reszty nie mają rozkładu normalnego.

\begin{Shaded}
\begin{Highlighting}[]
\FunctionTok{data.frame}\NormalTok{(}\StringTok{"Nazwa modelu"} \OtherTok{=} \FunctionTok{mapply}\NormalTok{(get\_name, }\DecValTok{1}\SpecialCharTok{:}\NormalTok{countM),}
           \StringTok{"p{-}value"} \OtherTok{=} \FunctionTok{mapply}\NormalTok{(}\ControlFlowTok{function}\NormalTok{(i)\{ }\FunctionTok{shapiro.test}\NormalTok{(}\FunctionTok{get\_model}\NormalTok{(i)}\SpecialCharTok{$}\NormalTok{residuals)}\SpecialCharTok{$}\NormalTok{p.value \}, }\DecValTok{1}\SpecialCharTok{:}\NormalTok{countM)) }
\end{Highlighting}
\end{Shaded}

\begin{verbatim}
##                                Nazwa.modelu      p.value
## 1                                      full 4.545244e-12
## 2                                  no sugar 4.497402e-13
## 3                                no alcohol 8.347294e-15
## 4                      no sugar, no alcohol 5.754722e-13
## 5                                   no acid 7.399467e-12
## 6                     no acid, no chlorides 7.659340e-12
## 7       no acid, no chlorides, no totalsulf 8.719048e-12
## 8                                       pca 4.545244e-12
## 9                                pca no pc6 1.544683e-11
## 10                       pca no pc6, no pc7 4.116944e-11
## 11                          cook no alcohol 5.437133e-12
## 12                             cook no acid 6.313852e-12
## 13               cook no acid, no chlorides 6.790857e-12
## 14 cook no acid, no chlorides, no totalsulf 7.470087e-12
## 15                  cook pca no pc6, no pc7 4.363168e-11
\end{verbatim}

Z powyższego testu wynika, że powinniśmy odrzucić hipotezę H0, czyli w
żadnym modelu reszty nie mają rozkładu normalnego. Rozbieżność między
testami a wykresem, może wynikać z dużej próbki, dla której niektóre
testy nie są adekwatne.

\subsubsection{Stałość wariancji}\label{staux142oux15bux107-wariancji}

W celu zbadania stałości wariancji zrobimy wykresy zależności zmiennych
resztowych od odpowiednich zmiennych dopasowanych.

\begin{Shaded}
\begin{Highlighting}[]
\FunctionTok{par}\NormalTok{(}\AttributeTok{mfrow =} \FunctionTok{c}\NormalTok{(}\FunctionTok{ceiling}\NormalTok{(countM }\SpecialCharTok{/} \DecValTok{4}\NormalTok{), }\DecValTok{4}\NormalTok{))}

\ControlFlowTok{for}\NormalTok{(i }\ControlFlowTok{in} \DecValTok{1}\SpecialCharTok{:}\NormalTok{countM)\{}
\NormalTok{  model }\OtherTok{\textless{}{-}} \FunctionTok{get\_model}\NormalTok{(i)}
\NormalTok{  p }\OtherTok{\textless{}{-}}\NormalTok{ (i }\SpecialCharTok{{-}} \DecValTok{1}\NormalTok{) }\SpecialCharTok{\%\%} \DecValTok{4}
\NormalTok{  q }\OtherTok{\textless{}{-}} \FunctionTok{ifelse}\NormalTok{((i }\SpecialCharTok{{-}} \DecValTok{1}\NormalTok{) }\SpecialCharTok{\%\%} \DecValTok{8} \SpecialCharTok{\textgreater{}=} \DecValTok{4}\NormalTok{, }\DecValTok{0}\NormalTok{, }\DecValTok{1}\NormalTok{)}
  \FunctionTok{par}\NormalTok{(}\AttributeTok{fig =} \FunctionTok{c}\NormalTok{(}\DecValTok{1}\SpecialCharTok{/}\DecValTok{4} \SpecialCharTok{*}\NormalTok{ p, }\DecValTok{1}\SpecialCharTok{/}\DecValTok{4} \SpecialCharTok{+} \DecValTok{1}\SpecialCharTok{/}\DecValTok{4} \SpecialCharTok{*}\NormalTok{ p, }\FloatTok{0.5} \SpecialCharTok{*}\NormalTok{ q, }\FloatTok{0.5} \SpecialCharTok{+} \FloatTok{0.5} \SpecialCharTok{*}\NormalTok{ q), }\AttributeTok{new =}\NormalTok{ (i }\SpecialCharTok{\%\%} \DecValTok{8} \SpecialCharTok{!=} \DecValTok{1}\NormalTok{))}
  \FunctionTok{plot}\NormalTok{(model}\SpecialCharTok{$}\NormalTok{fit, model}\SpecialCharTok{$}\NormalTok{res, }\AttributeTok{xlab =} \StringTok{"Zmienne dopasowane"}\NormalTok{, }\AttributeTok{ylab =} \StringTok{"Zmienne resztowe"}\NormalTok{, }\AttributeTok{main =} \FunctionTok{get\_name}\NormalTok{(i))}
  \FunctionTok{abline}\NormalTok{(}\AttributeTok{h =} \DecValTok{0}\NormalTok{, }\AttributeTok{col =} \StringTok{\textquotesingle{}red\textquotesingle{}}\NormalTok{)}
\NormalTok{\}}
\end{Highlighting}
\end{Shaded}

\includegraphics{Analiza-win_files/figure-latex/unnamed-chunk-35-1.pdf}
\includegraphics{Analiza-win_files/figure-latex/unnamed-chunk-35-2.pdf}

Przeprowadźmy jeszcze test Goldfeld-Quandt o stałości wariancji.

H0: reszty mają stałą wariancję,

H1: reszty nie mają stałej wariancji.

\begin{Shaded}
\begin{Highlighting}[]
\FunctionTok{data.frame}\NormalTok{(}\StringTok{"Nazwa modelu"} \OtherTok{=} \FunctionTok{mapply}\NormalTok{(get\_name, }\DecValTok{1}\SpecialCharTok{:}\NormalTok{countM),}
           \StringTok{"p{-}value"} \OtherTok{=} \FunctionTok{mapply}\NormalTok{(}\ControlFlowTok{function}\NormalTok{(i)\{ }\FunctionTok{gqtest}\NormalTok{(}\FunctionTok{get\_model}\NormalTok{(i))}\SpecialCharTok{$}\NormalTok{p.value \}, }\DecValTok{1}\SpecialCharTok{:}\NormalTok{countM)) }
\end{Highlighting}
\end{Shaded}

\begin{verbatim}
##                                Nazwa.modelu   p.value
## 1                                      full 0.2576003
## 2                                  no sugar 0.1993792
## 3                                no alcohol 0.6031103
## 4                      no sugar, no alcohol 0.7190410
## 5                                   no acid 0.2586811
## 6                     no acid, no chlorides 0.2610544
## 7       no acid, no chlorides, no totalsulf 0.2707617
## 8                                       pca 0.2576003
## 9                                pca no pc6 0.2791648
## 10                       pca no pc6, no pc7 0.3680088
## 11                          cook no alcohol 0.3027195
## 12                             cook no acid 0.2393948
## 13               cook no acid, no chlorides 0.2430445
## 14 cook no acid, no chlorides, no totalsulf 0.2473917
## 15                  cook pca no pc6, no pc7 0.3183690
\end{verbatim}

Widzimy, że w przypadku wszystkich modeli nie jesteśmy w stanie odrzucić
hipotezy H0 na poziomie istotności α = 0,05, dlatego wynika stąd, że
reszty mają stałą wariancję.

\subsubsection{Skorelowanie reszt}\label{skorelowanie-reszt}

W celu sprawdzenia, czy nasze reszty są skorelowane, posłużymy się
testem Durbina-Watsona:

H0: reszty nie są skorelowane, H1: istnieje korelacja reszt.

\begin{Shaded}
\begin{Highlighting}[]
\FunctionTok{data.frame}\NormalTok{(}\StringTok{"Nazwa modelu"} \OtherTok{=} \FunctionTok{mapply}\NormalTok{(get\_name, }\DecValTok{1}\SpecialCharTok{:}\NormalTok{countM),}
           \StringTok{"p{-}value"} \OtherTok{=} \FunctionTok{mapply}\NormalTok{(}\ControlFlowTok{function}\NormalTok{(i)\{ }\FunctionTok{durbinWatsonTest}\NormalTok{(}\FunctionTok{get\_model}\NormalTok{(i))}\SpecialCharTok{$}\NormalTok{p \}, }\DecValTok{1}\SpecialCharTok{:}\NormalTok{countM)) }
\end{Highlighting}
\end{Shaded}

\begin{verbatim}
##                                Nazwa.modelu p.value
## 1                                      full   0.196
## 2                                  no sugar   0.072
## 3                                no alcohol   0.394
## 4                      no sugar, no alcohol   0.096
## 5                                   no acid   0.186
## 6                     no acid, no chlorides   0.198
## 7       no acid, no chlorides, no totalsulf   0.174
## 8                                       pca   0.220
## 9                                pca no pc6   0.220
## 10                       pca no pc6, no pc7   0.094
## 11                          cook no alcohol   0.246
## 12                             cook no acid   0.198
## 13               cook no acid, no chlorides   0.194
## 14 cook no acid, no chlorides, no totalsulf   0.188
## 15                  cook pca no pc6, no pc7   0.122
\end{verbatim}

Na podstawie przeprowadzonego testu statystycznego, nie mamy podstaw do
odrzucenia hipotezy H0 o braku korelacji reszt dla każdego modelu
regresji liniowej.

\subsection{Współliniowość
regresorów}\label{wspuxf3ux142liniowoux15bux107-regresoruxf3w}

Aby przetestować współliniowość, posłużymy się statystyką Variance
Inflation Factor.

\begin{Shaded}
\begin{Highlighting}[]
\NormalTok{VIF }\OtherTok{=} \ControlFlowTok{function}\NormalTok{(i)\{}
  \FunctionTok{data.frame}\NormalTok{(}\StringTok{"Nazwa"} \OtherTok{=} \FunctionTok{get\_name}\NormalTok{(i), }\FunctionTok{t}\NormalTok{(}\FunctionTok{vif}\NormalTok{(}\FunctionTok{get\_model}\NormalTok{(i))))}
\NormalTok{\}}
\end{Highlighting}
\end{Shaded}

Wyliczając statystykę Variance Inflation Factor, a więc prostego testu
opartego na statystyce R2, który mierzy, jaka część wariancji estymatora
jest powodowana przez to, że zmienna j nie jest niezależna względem
pozostałych zmiennych objaśniających w modelu regresji, jesteśmy w
stanie określić współliniowość dla poszczególnych zmiennych.

W każdym modelu największą miarą charakteryzują się zmienne `density',
`residual.sugar' oraz `alcohol'. Możemy wywnioskować, że są to
najbardziej współliniowe zmienne, natomiast nie jest to zjawisko bardzo
mocno widoczne w naszym zbiorze danych. Modele bez tych parametrów są
mniej współliniowe.

\subsection{Miary dopasowania}\label{miary-dopasowania}

Przed wybraniem najlepszego modelu, spójrzmy jeszcze na miary
dopasowania modelu do danych.

\begin{Shaded}
\begin{Highlighting}[]
\FunctionTok{data.frame}\NormalTok{(}\StringTok{"Nazwa modelu"} \OtherTok{=} \FunctionTok{mapply}\NormalTok{(get\_name, }\DecValTok{1}\SpecialCharTok{:}\NormalTok{countM),}
           \StringTok{"Skorygowany współczynnik determinacji"}  \OtherTok{=} \FunctionTok{mapply}\NormalTok{(}\ControlFlowTok{function}\NormalTok{(index)\{}\FunctionTok{summary}\NormalTok{(}\FunctionTok{get\_model}\NormalTok{(index))}\SpecialCharTok{$}\NormalTok{adj.r.squared\}, }\DecValTok{1}\SpecialCharTok{:}\NormalTok{countM),}
           \StringTok{"Estymator wariancji błędów"} \OtherTok{=} \FunctionTok{mapply}\NormalTok{(}\ControlFlowTok{function}\NormalTok{(index)\{}\FunctionTok{summary}\NormalTok{(}\FunctionTok{get\_model}\NormalTok{(index))}\SpecialCharTok{$}\NormalTok{sigma\}, }\DecValTok{1}\SpecialCharTok{:}\NormalTok{countM),}
           \AttributeTok{check.names =} \ConstantTok{FALSE}\NormalTok{)}
\end{Highlighting}
\end{Shaded}

\begin{verbatim}
##                                Nazwa modelu
## 1                                      full
## 2                                  no sugar
## 3                                no alcohol
## 4                      no sugar, no alcohol
## 5                                   no acid
## 6                     no acid, no chlorides
## 7       no acid, no chlorides, no totalsulf
## 8                                       pca
## 9                                pca no pc6
## 10                       pca no pc6, no pc7
## 11                          cook no alcohol
## 12                             cook no acid
## 13               cook no acid, no chlorides
## 14 cook no acid, no chlorides, no totalsulf
## 15                  cook pca no pc6, no pc7
##    Skorygowany współczynnik determinacji Estymator wariancji błędów
## 1                              0.2884027                  0.7422036
## 2                              0.2732637                  0.7500571
## 3                              0.2693532                  0.7520724
## 4                              0.1570651                  0.8077983
## 5                              0.2883562                  0.7422278
## 6                              0.2886047                  0.7420983
## 7                              0.2883723                  0.7422195
## 8                              0.2884027                  0.7422036
## 9                              0.2877103                  0.7425646
## 10                             0.2789403                  0.7471220
## 11                             0.2872819                  0.7429377
## 12                             0.2916306                  0.7406677
## 13                             0.2917895                  0.7405846
## 14                             0.2919496                  0.7405009
## 15                             0.2847360                  0.7442635
\end{verbatim}

Widzimy, że gdybyśmy mieli wybierać najlepszy model na podstawie
skorygowanego współczynnika determinacji, to wybralibyśmy model `cook no
acid, no chlorides, no totalsulf'. Jednak wybór najlepszego modelu
opieramy na wyniku resztowej sumy kwadratów.

\subsection{Wybór najlepszego modelu na podstawie RSS oraz jego
test}\label{wybuxf3r-najlepszego-modelu-na-podstawie-rss-oraz-jego-test}

Stwórzmy funkcję, która pokaże odpowiednio: nazwę modelu; resztową sumę
kwadratów pomiędzy obserwacjami empirycznymi z próby walidacyjnej, a
przewidzianymi przez model regresji liniowej; odsetek poprawnych
klasyfikacji; odsetek poprawnych klasyfikacji różniących się o co
najwyżej jeden.

\begin{Shaded}
\begin{Highlighting}[]
\NormalTok{pca\_val }\OtherTok{\textless{}{-}} \FunctionTok{as.data.frame}\NormalTok{(}\FunctionTok{predict}\NormalTok{(pca, }\AttributeTok{newdata =}\NormalTok{ val[}\DecValTok{1}\SpecialCharTok{:}\DecValTok{11}\NormalTok{]))}
\NormalTok{pca\_tst }\OtherTok{\textless{}{-}} \FunctionTok{as.data.frame}\NormalTok{(}\FunctionTok{predict}\NormalTok{(pca, }\AttributeTok{newdata =}\NormalTok{ tst[}\DecValTok{1}\SpecialCharTok{:}\DecValTok{11}\NormalTok{]))}
\end{Highlighting}
\end{Shaded}

\begin{Shaded}
\begin{Highlighting}[]
\NormalTok{prediction\_summary }\OtherTok{\textless{}{-}} \FunctionTok{data.frame}\NormalTok{(}\FunctionTok{matrix}\NormalTok{(}\AttributeTok{ncol =} \DecValTok{4}\NormalTok{, }\AttributeTok{nrow =} \DecValTok{0}\NormalTok{, }\AttributeTok{dimnames =} \FunctionTok{list}\NormalTok{(}\ConstantTok{NULL}\NormalTok{, }\FunctionTok{c}\NormalTok{(}\StringTok{"Nazwa"}\NormalTok{, }\StringTok{"RSS"}\NormalTok{, }\StringTok{"Odsetek popr"}\NormalTok{,}\StringTok{"Odsetek róż"}\NormalTok{))))}
\end{Highlighting}
\end{Shaded}

\begin{Shaded}
\begin{Highlighting}[]
\NormalTok{make\_prediction }\OtherTok{=} \ControlFlowTok{function}\NormalTok{(index, non\_pca, iff\_pca)\{}
\NormalTok{  name  }\OtherTok{\textless{}{-}} \FunctionTok{get\_name}\NormalTok{(index)}
\NormalTok{  model }\OtherTok{\textless{}{-}} \FunctionTok{get\_model}\NormalTok{(index)}
  \ControlFlowTok{if}\NormalTok{(}\FunctionTok{is\_pca}\NormalTok{(index))\{}
\NormalTok{    prediction }\OtherTok{\textless{}{-}} \FunctionTok{round}\NormalTok{(}\FunctionTok{predict}\NormalTok{(model, iff\_pca))}
\NormalTok{  \}}\ControlFlowTok{else}\NormalTok{\{}
\NormalTok{    prediction }\OtherTok{\textless{}{-}} \FunctionTok{round}\NormalTok{(}\FunctionTok{predict}\NormalTok{(model, non\_pca[}\DecValTok{1}\SpecialCharTok{:}\DecValTok{11}\NormalTok{]))}
\NormalTok{  \}}
\NormalTok{  prediction\_summary[}\FunctionTok{nrow}\NormalTok{(prediction\_summary) }\SpecialCharTok{+} \DecValTok{1}\NormalTok{,] }\OtherTok{\textless{}\textless{}{-}} \FunctionTok{c}\NormalTok{(name,}
    \FunctionTok{sum}\NormalTok{((prediction }\SpecialCharTok{{-}}\NormalTok{ non\_pca[}\DecValTok{12}\NormalTok{])}\SpecialCharTok{\^{}}\DecValTok{2}\NormalTok{),}
    \FunctionTok{sum}\NormalTok{(non\_pca[}\DecValTok{12}\NormalTok{] }\SpecialCharTok{==}\NormalTok{ prediction, }\AttributeTok{na.rm =} \ConstantTok{TRUE}\NormalTok{) }\SpecialCharTok{/} \FunctionTok{nrow}\NormalTok{(non\_pca[}\DecValTok{12}\NormalTok{]) }\SpecialCharTok{*} \DecValTok{100}\NormalTok{,}
    \FunctionTok{sum}\NormalTok{(}\FunctionTok{abs}\NormalTok{(non\_pca[}\DecValTok{12}\NormalTok{] }\SpecialCharTok{{-}}\NormalTok{ prediction) }\SpecialCharTok{\textless{}=} \DecValTok{1}\NormalTok{)}\SpecialCharTok{/} \FunctionTok{nrow}\NormalTok{(non\_pca[}\DecValTok{12}\NormalTok{]) }\SpecialCharTok{*} \DecValTok{100}\NormalTok{)}
\NormalTok{\}}
\end{Highlighting}
\end{Shaded}

\begin{Shaded}
\begin{Highlighting}[]
\ControlFlowTok{for}\NormalTok{(i }\ControlFlowTok{in} \DecValTok{1}\SpecialCharTok{:}\NormalTok{countM)\{ }\FunctionTok{make\_prediction}\NormalTok{(i, val, pca\_val) \}}
\end{Highlighting}
\end{Shaded}

\begin{Shaded}
\begin{Highlighting}[]
\NormalTok{prediction\_summary}
\end{Highlighting}
\end{Shaded}

\begin{verbatim}
##                                       Nazwa RSS     Odsetek.popr
## 1                                      full 809 50.9031198686371
## 2                                  no sugar 806 51.3957307060755
## 3                                no alcohol 794 51.3957307060755
## 4                      no sugar, no alcohol 927 47.6190476190476
## 5                                   no acid 797 51.1494252873563
## 6                     no acid, no chlorides 798 51.3136288998358
## 7       no acid, no chlorides, no totalsulf 788 51.6420361247947
## 8                                       pca 809 50.9031198686371
## 9                                pca no pc6 790 51.4778325123153
## 10                       pca no pc6, no pc7 772 51.6420361247947
## 11                          cook no alcohol 789  51.559934318555
## 12                             cook no acid 792  51.559934318555
## 13               cook no acid, no chlorides 799 51.2315270935961
## 14 cook no acid, no chlorides, no totalsulf 785 51.3957307060755
## 15                  cook pca no pc6, no pc7 768 52.2167487684729
##         Odsetek.róż
## 1  94.4991789819376
## 2  94.4170771756979
## 3  94.7454844006568
## 4  92.7750410509031
## 5  94.7454844006568
## 6  94.6633825944171
## 7  94.8275862068966
## 8  94.4991789819376
## 9  94.8275862068966
## 10 95.4022988505747
## 11 94.8275862068966
## 12 94.7454844006568
## 13 94.6633825944171
## 14  94.991789819376
## 15  95.320197044335
\end{verbatim}

Poprzez utworzoną tabelę, dostrzegamy że modele o najmniejszej resztowej
sumie kwadratów, a więc nasze najlepsze modele na tej podstawie, to `pca
no pc6' oraz `pca no pc6, no pc7'. Jednak model `pca no 6' ma większy
odsetek poprawnych klasyfikacji, dlatego to właśnie on zostanie wybrany
do dalszych testów.

Przetestujmy teraz w takim razie nasz najlepszy model.

\begin{Shaded}
\begin{Highlighting}[]
\NormalTok{prediction\_summary }\OtherTok{=} \ConstantTok{NULL}
\NormalTok{prediction\_summary }\OtherTok{\textless{}{-}} \FunctionTok{data.frame}\NormalTok{(}\FunctionTok{matrix}\NormalTok{(}\AttributeTok{ncol =} \DecValTok{4}\NormalTok{, }\AttributeTok{nrow =} \DecValTok{0}\NormalTok{, }\AttributeTok{dimnames =} \FunctionTok{list}\NormalTok{(}\ConstantTok{NULL}\NormalTok{, }\FunctionTok{c}\NormalTok{(}\StringTok{"Nazwa"}\NormalTok{, }\StringTok{"RSS"}\NormalTok{, }\StringTok{"odsetek popr. kl"}\NormalTok{,}\StringTok{"Odsetek kl. o 1"}\NormalTok{))))}

\FunctionTok{make\_prediction}\NormalTok{(}\DecValTok{9}\NormalTok{, tst, pca\_tst)}

\NormalTok{prediction\_summary}
\end{Highlighting}
\end{Shaded}

\begin{verbatim}
##        Nazwa RSS odsetek.popr..kl  Odsetek.kl..o.1
## 1 pca no pc6 827 51.1092851273624 95.0698438783895
\end{verbatim}

Widzimy, że nasz model nie jest wybitny, natomiast dobrze poradził sobie
z próbą testową, przewidując ponad połowę poprawnych klasyfikacji.
Możemy z tego wywnioskować, że jest to najlepszy utworzony przez nas
model regresji liniowej, z drugiej jednak strony sama regresja liniowa
jest obarczona dużym błędem w przewidywaniach.

\section{Model proporcjonalnych
szans}\label{model-proporcjonalnych-szans}

W celu zobrazowania zmiany zmiennej quality na zmienną jakościową
utworzymy model proporcjonalnych szans:

\begin{Shaded}
\begin{Highlighting}[]
\NormalTok{wine1 }\OtherTok{\textless{}{-}}\NormalTok{ white\_numeric}
\NormalTok{wine1}\SpecialCharTok{$}\NormalTok{quality }\OtherTok{\textless{}{-}} \FunctionTok{factor}\NormalTok{(wine1}\SpecialCharTok{$}\NormalTok{quality)}
\end{Highlighting}
\end{Shaded}

Oraz podzielimy nasz zbiór na 3 podzbiory, jak wcześniej.

\begin{Shaded}
\begin{Highlighting}[]
\NormalTok{lrn2 }\OtherTok{\textless{}{-}}\NormalTok{ wine1[I\_l,]}
\NormalTok{val2 }\OtherTok{\textless{}{-}}\NormalTok{ wine1[I\_v,]}
\NormalTok{tst2 }\OtherTok{\textless{}{-}}\NormalTok{ wine1[I\_t,]}
\end{Highlighting}
\end{Shaded}

Utworzony model wygląda następująco:

\begin{Shaded}
\begin{Highlighting}[]
\NormalTok{g.plr }\OtherTok{\textless{}{-}} \FunctionTok{polr}\NormalTok{(quality }\SpecialCharTok{\textasciitilde{}}\NormalTok{ ., }\AttributeTok{data =}\NormalTok{ lrn2)}
\NormalTok{g.plr}
\end{Highlighting}
\end{Shaded}

\begin{verbatim}
## Call:
## polr(formula = quality ~ ., data = lrn2)
## 
## Coefficients:
##        fixed.acidity     volatile.acidity          citric.acid 
##         1.171114e-01        -5.037121e+00         5.143532e-01 
##       residual.sugar            chlorides  free.sulfur.dioxide 
##         1.874274e-01         3.926045e-01         1.429007e-02 
## total.sulfur.dioxide              density                   pH 
##        -2.226968e-03        -3.128114e+02         1.866293e+00 
##            sulphates              alcohol 
##         1.555264e+00         6.349557e-01 
## 
## Intercepts:
##       3|4       4|5       5|6       6|7       7|8       8|9 
## -302.8883 -300.5129 -297.5061 -294.8270 -292.6198 1869.6047 
## 
## Residual Deviance: 5350.84 
## AIC: 5384.84
\end{verbatim}

Przyjrzyjmy się teraz jego podsumowaniu.

\begin{Shaded}
\begin{Highlighting}[]
\FunctionTok{summary}\NormalTok{(g.plr)}
\end{Highlighting}
\end{Shaded}

\begin{verbatim}
## 
## Ponowne dopasowywanie aby uzyskać hesjan
\end{verbatim}

\begin{verbatim}
## Call:
## polr(formula = quality ~ ., data = lrn2)
## 
## Coefficients:
##                           Value Std. Error   t value
## fixed.acidity         1.171e-01   0.053719    2.1801
## volatile.acidity     -5.037e+00   0.435293  -11.5718
## citric.acid           5.144e-01   0.334881    1.5359
## residual.sugar        1.874e-01   0.009756   19.2125
## chlorides             3.926e-01   1.933109    0.2031
## free.sulfur.dioxide   1.429e-02   0.003169    4.5100
## total.sulfur.dioxide -2.227e-03   0.001344   -1.6567
## density              -3.128e+02   0.658271 -475.2012
## pH                    1.866e+00   0.299841    6.2243
## sulphates             1.555e+00   0.348255    4.4659
## alcohol               6.350e-01   0.044907   14.1393
## 
## Intercepts:
##     Value     Std. Error t value  
## 3|4 -302.8883    0.6702  -451.9593
## 4|5 -300.5129    0.6671  -450.4928
## 5|6 -297.5061    0.6739  -441.4666
## 6|7 -294.8270    0.6889  -427.9466
## 7|8 -292.6198    0.7016  -417.0619
## 8|9 1869.6047    0.7016  2664.6892
## 
## Residual Deviance: 5350.84 
## AIC: 5384.84
\end{verbatim}

Zastosujemy funkcję predict() do przewidywania wartości dla tego modelu.

\begin{Shaded}
\begin{Highlighting}[]
\NormalTok{pr\_log1 }\OtherTok{\textless{}{-}} \FunctionTok{predict}\NormalTok{(g.plr, val2[,}\DecValTok{1}\SpecialCharTok{:}\DecValTok{11}\NormalTok{], }\AttributeTok{type=}\StringTok{"class"}\NormalTok{)}

\NormalTok{val21 }\OtherTok{\textless{}{-}} \FunctionTok{as.numeric}\NormalTok{(val2}\SpecialCharTok{$}\NormalTok{quality)}
\NormalTok{pr\_log11 }\OtherTok{\textless{}{-}} \FunctionTok{as.numeric}\NormalTok{(pr\_log1)}
\end{Highlighting}
\end{Shaded}

Oraz obliczmy resztową sumę kwadratów.

\begin{Shaded}
\begin{Highlighting}[]
\FunctionTok{sum}\NormalTok{((pr\_log11}\SpecialCharTok{{-}}\NormalTok{val21)}\SpecialCharTok{\^{}}\DecValTok{2}\NormalTok{)}
\end{Highlighting}
\end{Shaded}

\begin{verbatim}
## [1] 805
\end{verbatim}

Widzimy, że model jest przeciętny, więc ulepszymy go. Uprościmy model za
pomocą funkcji step.

\begin{Shaded}
\begin{Highlighting}[]
\NormalTok{o\_lr }\OtherTok{=} \FunctionTok{step}\NormalTok{(g.plr)}
\end{Highlighting}
\end{Shaded}

\begin{verbatim}
## Start:  AIC=5384.84
## quality ~ fixed.acidity + volatile.acidity + citric.acid + residual.sugar + 
##     chlorides + free.sulfur.dioxide + total.sulfur.dioxide + 
##     density + pH + sulphates + alcohol
## 
##                        Df    AIC
## - chlorides             1 5382.9
## <none>                    5384.8
## - fixed.acidity         1 5384.9
## - citric.acid           1 5385.2
## - total.sulfur.dioxide  1 5385.4
## - sulphates             1 5401.0
## - free.sulfur.dioxide   1 5402.8
## - density               1 5403.2
## - pH                    1 5405.8
## - alcohol               1 5418.9
## - residual.sugar        1 5429.9
## - volatile.acidity      1 5518.5
## 
## Step:  AIC=5382.88
## quality ~ fixed.acidity + volatile.acidity + citric.acid + residual.sugar + 
##     free.sulfur.dioxide + total.sulfur.dioxide + density + pH + 
##     sulphates + alcohol
## 
##                        Df    AIC
## <none>                    5382.9
## - citric.acid           1 5383.4
## - total.sulfur.dioxide  1 5383.5
## - fixed.acidity         1 5386.3
## - sulphates             1 5399.1
## - free.sulfur.dioxide   1 5401.0
## - density               1 5401.7
## - pH                    1 5407.8
## - alcohol               1 5417.5
## - residual.sugar        1 5430.1
## - volatile.acidity      1 5521.4
\end{verbatim}

Utworzony model ma mniej zmiennych zależnych w celu zmniejszenia AIC,
która estymuje liczbę utraconych danych przez dany model. Im mniej
danych model utracił, tym lepszej jest jakości. Innymi słowy AIC określa
ryzyko przeszacowania oraz niedoszacowania modelu.

Spójrzmy teraz na podsumowanie.

\begin{Shaded}
\begin{Highlighting}[]
\FunctionTok{summary}\NormalTok{(o\_lr)}
\end{Highlighting}
\end{Shaded}

\begin{verbatim}
## 
## Ponowne dopasowywanie aby uzyskać hesjan
\end{verbatim}

\begin{verbatim}
## Call:
## polr(formula = quality ~ fixed.acidity + volatile.acidity + citric.acid + 
##     residual.sugar + free.sulfur.dioxide + total.sulfur.dioxide + 
##     density + pH + sulphates + alcohol, data = lrn2)
## 
## Coefficients:
##                           Value Std. Error  t value
## fixed.acidity         1.137e-01   0.053480    2.126
## volatile.acidity     -5.028e+00   0.432303  -11.631
## citric.acid           5.254e-01   0.329697    1.594
## residual.sugar        1.859e-01   0.009624   19.321
## free.sulfur.dioxide   1.432e-02   0.003168    4.520
## total.sulfur.dioxide -2.232e-03   0.001344   -1.661
## density              -3.095e+02   0.639628 -483.932
## pH                    1.851e+00   0.298762    6.194
## sulphates             1.550e+00   0.348192    4.451
## alcohol               6.359e-01   0.042742   14.878
## 
## Intercepts:
##     Value     Std. Error t value  
## 3|4 -299.7198    0.6485  -462.1784
## 4|5 -297.3445    0.6455  -460.6453
## 5|6 -294.3376    0.6528  -450.8736
## 6|7 -291.6586    0.6685  -436.2950
## 7|8 -289.4515    0.6816  -424.6866
## 8|9 1800.4887    0.6816  2641.6981
## 
## Residual Deviance: 5350.88 
## AIC: 5382.88
\end{verbatim}

\begin{Shaded}
\begin{Highlighting}[]
\FunctionTok{anova}\NormalTok{(g.plr,o\_lr)}
\end{Highlighting}
\end{Shaded}

\begin{verbatim}
## Likelihood ratio tests of ordinal regression models
## 
## Response: quality
##                                                                                                                                                           Model
## 1             fixed.acidity + volatile.acidity + citric.acid + residual.sugar + free.sulfur.dioxide + total.sulfur.dioxide + density + pH + sulphates + alcohol
## 2 fixed.acidity + volatile.acidity + citric.acid + residual.sugar + chlorides + free.sulfur.dioxide + total.sulfur.dioxide + density + pH + sulphates + alcohol
##   Resid. df Resid. Dev   Test    Df   LR stat.   Pr(Chi)
## 1      2419    5350.88                                  
## 2      2418    5350.84 1 vs 2     1 0.03975412 0.8419621
\end{verbatim}

Widzimy, że mniejszy model jest jak najbardziej zasadny. Policzmy
resztową sumę kwadratów.

\begin{Shaded}
\begin{Highlighting}[]
\NormalTok{pr\_log2}\OtherTok{\textless{}{-}}\FunctionTok{predict}\NormalTok{(o\_lr, val2[,}\DecValTok{1}\SpecialCharTok{:}\DecValTok{11}\NormalTok{],}\AttributeTok{type=}\StringTok{"class"}\NormalTok{)}

\NormalTok{val21}\OtherTok{\textless{}{-}} \FunctionTok{as.numeric}\NormalTok{(val2}\SpecialCharTok{$}\NormalTok{quality)}
\NormalTok{pr\_log12 }\OtherTok{\textless{}{-}} \FunctionTok{as.numeric}\NormalTok{(pr\_log2)}
\FunctionTok{sum}\NormalTok{((pr\_log12}\SpecialCharTok{{-}}\NormalTok{val21)}\SpecialCharTok{\^{}}\DecValTok{2}\NormalTok{)}
\end{Highlighting}
\end{Shaded}

\begin{verbatim}
## [1] 807
\end{verbatim}

Resztowa suma kwadratów jest większa, wiec model nie jest tak dobry jak
wyjściowy, klasyfikując modele na podstawie tego kryterium, dlatego
żaden z tych modeli nie jest lepszy w porównaniu z najlepszym modelem
regresji liniowej.

\section{Podsumowanie}\label{podsumowanie}

W projekcie utworzone zostało 17 modeli regresji liniowych oraz 2 modele
proporcjonalnych szans.

Wśród wszystkich modeli najmniejszą wartość resztowej sumy kwadratów
osiągnął model regresji liniowej, który został ostatecznie przetestowany
dla zbioru testowego. Na jego podstawie określiliśmy, że model regresji
liniowej jest całkiem dokładny, ponieważ próba testowa dała nam bardzo
dobry wynik.

Można oczywiście tworzyć następne modele, wyrzucać kolejne zmienne
odstające i je analizować, jednak my postanowiliśmy ograniczyć się tylko
do tych kilku modeli. Natomiast widać spory problem w tym, że regresja
liniowa nie jest idealnym modelem dla naszego zbioru danych.

\end{document}
